\documentclass[11pt,fleqn]{article}
\usepackage[cm]{fullpage}
%%AVC PACKAGES
\usepackage{avcgreek}
\usepackage{avcfonts}
\usepackage{avcmath}
\usepackage[numberby=section]{avcthm} % 
\usepackage{qcmacros}
\usepackage{goldstone}
%%MACROS FOR THIS DOCUMENT
\numberwithin{equation}{section}
\usepackage{titlesec}
\titleformat{\section}{\Large\bfseries\mathversion{bold}}{\thesection.}{5pt}{}
% remove header from TOC
\makeatletter
\renewcommand\tableofcontents{%
  \@starttoc{toc}%
}
\makeatother
\newcommand{\resolventline}[2][1]{
  \tikz[overlay]{
      \draw[thick,flexdotted] (0,-1ex) to ++(0,#1*4.5ex) node[above,inner sep=1pt] {#2};
  }
}

%%%DOCUMENT%%%
\begin{document}

%%TENSORS
\section{Perturbation theory and the linked diagram theorem}\label{sec:rspt}

\begin{dfn}\label{dfn:model-hamiltonian}
\thmtitle{Model problem}
In perturbation theory, the \textit{model Hamiltonian} is an operator $H_0\approx H_e$ which approximates the full Hamiltonian and has eigenfunctions spanning the complete $n$-particle Fock space.
An obvious choice in electronic structure is the diagonal one-particle component of $H_c$, which is diagonal in the determinant basis.
\begin{align*}
&&
  H_0
  \F_{i_1\cd i_k}^{a_1\cd a_k}
=
  \mc{E}_{i_1\cd i_k}^{a_1\cd a_k}
  \F_{i_1\cd i_k}^{a_1\cd a_k}
&&
  H_c
=
  H_0
+
  V_c
&&
  H_0
\equiv
  f_p^p\tl{a}^p_p
&&
  \mc{E}_{i_1\cd i_k}^{a_1\cd a_k}
\equiv
\pr{
  f_{a_1}^{a_1}
+
  \cd
+
  f_{a_k}^{a_k}
-
  f_{i_1}^{i_1}
-
  \cd
-
  f_{i_k}^{i_k}
}
\end{align*}
where the perurbation is
$
  V_c
\equiv
  f_p^q(1-\d_p^q)
  \tl{a}^p_q
+
  \tfr{1}{4}
  \ol{g}_{pq}^{rs}
  \tl{a}^{pq}_{rs}
$.
In this context, the reference determinant $\F$ is termed the \textit{model function}.
Note that we are playing fast and loose with Einstein summation here: the $i$ and $a$ indices are not summed over, but the $p$ index in $H_0$ is.
The eigenvalue can be determined by noting that $a^p_p\F_k=n_p^k\F_k$ where $n_p^k$ is the occupation number of $\y_p$ in $\F_k$.
Therefore $\tl{a}^p_p=a^p_p-\tl{a}^{p^\hole}_{p^\hole}=a^p_p - n_p$ which implies that $\tl{a}^p_p\F_k=(n_p^k-n_p)\F_k$.
It then follows that
$
  \sum_p
  f_p^p
  \tl{a}^p_p
  \F_k
=
(
  \sum_{p\in \F_k}
  f_p^p
-
  \sum_{p\in \F}
  f_p^p
)
\F_k
$,
which leads to the expression above for $\mc{E}_{i_1\cd i_k}^{a_1\cd a_k}$.
\end{dfn}


\begin{dfn}\label{dfn:model-function}
\thmtitle{Model space projection operators}
The projection $P=\kt{\F}\br{\F}$ onto the model function is termed the \textit{model space projection operator}, and its orthogonal complement $Q=1-P=\sum_k(\fr{1}{k!})^2\sum_{\substack{a_1\cd a_k\\i_1\cd i_k}}\kt{\F_{i_1\cd i_k}^{a_1\cd a_k}}\br{\F_{i_1\cd i_k}^{a_1\cd a_k}}$ is the \textit{orthogonal space projection operator}.
Note that $P^2=Q^2=1$ and $PQ=QP=0$ are necessary consequences of the fact that $P$ and $Q$ are projection operators for orthogonal subspaces, and note that $P+Q=1$.
Assuming \textit{intermediate normalization}, where we normalize the wavefunction to satisfy $\ip{\F|\Y}=1$ rather than $\ip{\Y|\Y}=1$, the model space projection operator takes the wavefunction into our model function, $P\Y=\F\ip{\F|\Y}=\F$.
\end{dfn}

\begin{dfn}
\thmtitle{Resolvent}
Let $R_0$ be minus\footnote{The annoying sign factor is required for consistency with the standard definition $R_0\equiv(E_0-H_0)^{-1}Q$.  Since we have already subtracted off $E_0$, we have $R_0=(-H_0)^{-1}Q$.  This also results in a more convenient sign rule for the bracketing theorem.} the inverse of $H_0$ in the orthogonal space, so that $-R_0H_0=Q$.
The operator $R_0$ is termed the \textit{resolvent}.
Explicitly, we can apply resolution of the identity in the orthogonal space to get
{\footnotesize
\begin{align*}
  R_0
=
  (-H_0)^{-1}Q
=
  \sum_k
  \pr{\tfr{1}{k!}}^2
  \sum_{\substack{a_1\cd a_k\\i_1\cd i_k}}
  (-H_0)^{-1}
  \kt{\F_{i_1\cd i_k}^{a_1\cd a_k}}
  \br{\F_{i_1\cd i_k}^{a_1\cd a_k}}
=
  \sum_k
  \pr{\tfr{1}{k!}}^2
  \sum_{\substack{a_1\cd a_k\\i_1\cd i_k}}
  \fr{
    \kt{\F_{i_1\cd i_k}^{a_1\cd a_k}}
    \br{\F_{i_1\cd i_k}^{a_1\cd a_k}}
  }{
    \mc{D}_{i_1\cd i_k}^{a_1\cd a_k}
  }
&&
&&
&&
  \mc{D}_{i_1\cd i_k}^{a_1\cd a_k}
\equiv
-
  \mc{E}_{i_1\cd i_k}^{a_1\cd a_k}
\end{align*}}%
where we recognize that $H_0^{-1}$ does not exist outside of the model space because $H_0\F=0\implies H_0P=0$.
Note that $R_0$ simply acts as the null operator outside of the orthogonal space, so that $R_0Q=R_0$ and $R_0P=0$.
\end{dfn}

\begin{dfn}
\thmtitle{Rayleigh-Schr\"odinger perturbation theory}
Projecting the Schr\"odinger equation by $R_0$, recognizing that
$R_0H_0=-Q\implies R_0H_0\Y=-Q\Y=(P-1)\Y=\F-\Y$,
we obtain the following recursive equation for $\Y$.
\begin{align*}
  R_0
  (H_0 + V_c)\Y
=
  \F
-
  \Y
+
  R_0V_c\Y
=
  E_cR_0\Y
\implies
  \Y
=
  \F
+
  R_0
  (V_c - E_c)
  \Y
\end{align*}
Straightforward induction leads to
$
  \Y
=
  \sum_{k=0}^n
  (R_0(V_c - E_c))^k\F
+
  (R_0(V_c - E_c))^{n+1}\Y
$.
Noting that $H_0\F=0$ and $\ip{\F|\Y}=1$, projecting the Schr\"odinger equation by $\F$ gives an expression for the correlation energy: $E_c=\ip{\F|V_c|\Y}$.
Assuming the series converges, this leads to the following expressions for the wavefunction and the correlation energy.
\begin{align}\label{eq:rspt-equations-form-1}
  \Y
=
  \sum_{k=0}^{\infty}
  (R_0(V_c - E_c))^k\F
&&
  E_c
=
  \sum_{k=0}^{\infty}
  \ip{\F|V_c(R_0(V_c - E_c))^k|\F}
\end{align}
Introducing a perturbation parameter $V_c\mapsto \la V_c$ and expanding the wavefunction and energy in a Taylor series
\begin{align*}
  \Y(\la)
=
  \sum_k
  \fr{1}{k!}\,\la^k\pr{\pd{^k\Y}{\la^k}}_{\la=0}
\equiv
  \sum_k
  \la^k \Y\ord{k}
&&
  E_c(\la)
=
  \sum_k
  \fr{1}{k!}\,\la^k\pr{\pd{^kE_c}{\la^k}}_{\la=0}
\equiv
  \sum_k
  \la^k
  E_c\ord{k}
\end{align*}
allows us to separate \Cref{eq:rspt-equations-form-1} in orders of $\la$.
The first-order energy contribution vanishes
$\la E_c\ord{1}=\la\ip{\F|V_c|\F}=0$
since $V_c$ is composed of $\F$-normal operators.
The first order wavefunction contribution is
$
  \la
  \Y\ord{1}
=
  \la
  R_0
  (V_c - E_c\ord{1})\F
=
  \la
  R_0
  V_c
  \F
$,
which can be directly evaluated using Wick's theorem and $\F$ normal ordering
\begin{align*}
  \Y\ord{1}
=
  R_0
  V_c
  \F
=&\
  \sum_k
  \pr{\fr{1}{k!}}^2
  \sum_{\substack{a_1\cd a_k\\i_1\cd i_k}}
  \kt{\F_{i_1\cd i_k}^{a_1\cd a_k}}
  \fr{
    \ip{\F|\tl{a}^{i_1\cd i_k}_{a_1\cd a_k}V_c|\F}
  }{
    \mc{D}_{i_1\cd i_k}^{a_1\cd a_k}
  }
\\=&\
  \sum_{ia}
  \kt{\F_i^a}
  \fr{
    \sum_{pq}
    f_p^q(1-\d_p^q)
    \ip{\F|\gno{\tl{a}_{a^\ptcl}^{i^\hole}\tl{a}^{p^\ptcl}_{q^\hole}}|\F}
  }{
    f_i^i
  -
    f_a^a
  }
+
  \fr{1}{4}
  \sum_{ijab}
  \kt{\F_{ij}^{ab}}
  \fr{
    \fr{1}{4}
    \sum_{pqrs}
    \ol{g}_{pq}^{rs}
    P^{(p/q)}_{(r/s)}
    \ip{\F|\gno{\tl{a}_{a^\ptcl b^{\ptcl\ptcl}}^{i^\hole j^{\hole\hole}}\tl{a}^{p^\ptcl q^{\ptcl\ptcl}}_{r^\hole s^{\hole\hole}}}|\F}
  }{
    f_i^i
  +
    f_j^j
  -
    f_a^a
  -
    f_b^b
  }
\\=&\
  \sum_{ia}
  \kt{\F_i^a}
  \fr{
    f_a^i
  }{
    f_i^i
  -
    f_a^a
  }
+
  \fr{1}{4}
  \sum_{ijab}
  \kt{\F_{ij}^{ab}}
  \fr{
    \ol{g}_{ab}^{ij}
  }{
    f_i^i
  +
    f_j^j
  -
    f_a^a
  -
    f_b^b
  }
\end{align*}
where we have recognized that only singly and doubly excited determinants can fully contract $V_c$.
The second-order energy contribution, $\la^2 E_c\ord{2}=\la^2\ip{\F|V_c|\Y\ord{1}}$, can be evaluated from our expression for $\Y\ord{1}$.
\begin{align*}
  E\ord{2}
=
  \sum_{ia}
  \ip{\F|V_c\tl{a}^a_i|\F}
  \fr{
    f_a^i
  }{
    f_i^i
  -
    f_a^a
  }
+
  \fr{1}{4}
  \sum_{ijab}
  \ip{\F|V_c\tl{a}^{ab}_{ij}|\F}
  \fr{
    \ol{g}_{ab}^{ij}
  }{
    f_i^i
  +
    f_j^j
  -
    f_a^a
  -
    f_b^b
  }
=
  \sum_{ia}
  \fr{
    f_i^a
    f_a^i
  }{
    f_i^i
  -
    f_a^a
  }
+
  \fr{1}{4}
  \sum_{ijab}
  \fr{
    \ol{g}_{ij}^{ab}
    \ol{g}_{ab}^{ij}
  }{
    f_i^i
  +
    f_j^j
  -
    f_a^a
  -
    f_b^b
  }
\end{align*}
The one-particle contributions involving $f_i^a$ are present only for ROHF references, since $f_i^a=0$ for canonical Hartree-Fock orbitals.
The second order wavefunction contribution is
$
  \la^2
  \Y\ord{2}
=
-
  \la^2
  E_c\ord{2}
  R_0\F
+
  \la^2
  R_0(V_c - E_c\ord{1})R_0(V_c - E_c\ord{1})\F
=
  \la^2
  R_0V_cR_0V_c\F
$
since $R_0\F=0$ and $E_c\ord{1}=0$.
The third order energy can be then obtained from $\Y\ord{2}$ as
$\la^3 E_c\ord{3} = \la^3\ip{\F|V_c|\Y\ord{2}}$.
In this manner, one can (in principle) solve the Schr\"odinger equation recursively by alternately evaluating the wavefunction and energy contributions at increasing orders in the perturbation parameter.
\end{dfn}

\begin{drv}
Writing the RSPT wavefunction equation as $\Y=\sum_{k=0}^{\infty}(R_0V_c - R_0 E_c)^k\F$, note that, if $R_0$ and $V_c$ were to commute, we could apply the binomial theorem to write the right-hand side as $\sum_{p=0}^k{k\choose p}(-)^p(R_0V_c)^{k-p}(R_0E_c)^p$.
Since they don't commute, we have to settle for the following generalization of the binomial expansion
\begin{align*}
  (R_0E_c - R_0V_c)^k
=
  \sum_{p=0}^k
  (-)^p
  \{R_0E_c\}^p\mr{insert}\{R_0V_c\}^{k-p}
\end{align*}
where $\{B_1,\ld,B_p\}\mr{insert}\{A\}^{k-p}$ denotes the sum over all ${k\choose p}$ possible ways of inserting $k-p$ copies of $A$ into the product $B_1\cd B_p$.
For example,
$
  \{B_1,B_2\}\mr{insert}\{A\}^2
$
evaluates to
$
  AAB_1B_2
+
  AB_1AB_2
+
  AB_1B_2A
+
  B_1AAB_2
+
  B_1AB_2A
+
  B_1B_2AA
$.
This allows the wavefunction expansion to be easily grouped by orders
\begin{align*}
  \Y
=&\
  \sum_{k=0}^{\infty}
  \sum_{p=0}^k
  (-)^p
  \{R_0E_c\}^p
  \mr{insert}\{R_0V_c\}^{k-p}
  \F
\\=&\
  \sum_{n=0}^{\infty}
  \sum_{(n_1,n_2)}^{\mc{C}_2(n)\cup\{(0,n)\}}
  \sum^{\mc{C}(n_1)}_{(r_1,\ld,r_m)}
  (-)^m
  \{R_0E_c\ord{r_1},\ld,R_0E_c\ord{r_m}\}
  \mr{insert}
  \{R_0V_c\}^{n_2}
  \F
=
  \sum_{n=0}^\infty
  \Y\ord{n}
\end{align*}
where $\mc{C}(n)$ denotes the set of integer compositions of $n$, i.e. all ordered tuples $(r_1,\ld,r_m)$ of strictly positive integers that add up to $n$.
$\mc{C}_k(n)\subset\mc{C}(n)$ is the set of $k$-tuple integer compositions of $n$, i.e. all $(r_1,\ld,r_k)$ of fixed length $k$ such that $r_1+\cd+r_k=n$.
The rearrangement follows from the fact that all possible terms of the form
$
  (-)^k
  \{R_0E_c\ord{n_1},\ld,R_0E_c\ord{n_k}\}
  \mr{insert}
  \{R_0V_c\}^{n_{k+1}}
  \F
$
contribute to the sum, and the composition sums group these into all possible terms of this form that are of a given order $n$ in the perturbation parameter $\la$.
Note that we have appended the tuple $(0,n)$ to our sum over $\mc{C}_2(n)$ but not $(n,0)$ since $R_0$ acting directly on $\F$ gives $0$.
These results are summarized in \Cref{lem:energy-substitution-lemma}.
\end{drv}

\begin{lem}\label{lem:energy-substitution-lemma}
\thmtitle{The Energy Substitution Lemma}
\thmstatement{
The $n\eth$-order contribution to the wavefunction is given by
\begin{align*}
  \Y\ord{n}
=
  (R_0V_c)^n\F
+
  \sum_{(n_1,n_2)}^{\mc{C}_2(n)}
  \sum^{\mc{C}(n_1)}_{(r_1,\ld,r_m)}
  (-)^m
  \{R_0E_c\ord{r_1},\ld,R_0E_c\ord{r_m}\}
  \mr{insert}
  \{R_0V_c\}^{n_2}
  \F
\end{align*}
which can be evaluated as the sum of a \emph{principal term}, $(R_0V_c)^n\F$, plus all possible $m$-tuple substitutions of adjacent factors $(R_0V_c)^{r_k}$ in the principal term by $R_0E_c\ord{r_k}$ times a sign factor $(-)^m$.
}
\end{lem}

\begin{ex}
Using the energy substitution lemma, we can directly write down the first few wavefunction contributions
\begin{align*}
  \Y\ord{1}
=&\
  R_0V_c\F
\\
  \Y\ord{2}
=&\
  R_0V_cR_0V_c\F
-
  R_0E_c\ord{1}R_0V_c\F
-
  R_0V_cR_0E_c\ord{1}\F
\\
  \Y\ord{3}
=&\
  R_0V_cR_0V_cR_0V_c\F
-
  R_0E_c\ord{2}R_0V_c\F
-
  R_0V_cR_0E_c\ord{2}\F
+
  R_0E_c\ord{1}R_0E_c\ord{1}R_0V_c\F
+
  R_0E_c\ord{1}R_0V_cR_0E_c\ord{1}\F
\\&\
+
  R_0V_cR_0E_c\ord{1}R_0E_c\ord{1}\F
-
  R_0E_c\ord{1}R_0V_cR_0V_c\F
-
  R_0V_cR_0E_c\ord{1}R_0V_c\F
-
  R_0V_cR_0V_cR_0E_c\ord{1}\F
\end{align*}
where we have directly evaluated the formula of \Cref{lem:energy-substitution-lemma} without simplifying.
These expressions can be simplified by recognizing that $E_c\ord{1}=0$ and that any term with an energy factor next to $\F$ vanishes since $R_0\F=0$.
Omitting these terms, the wavefunction contributions can be simplified as follows.
\begin{align*}
  \Y\ord{1}
=&\
  R_0V_c\F
\\
  \Y\ord{2}
=&\
  R_0V_cR_0V_c\F
\\
  \Y\ord{3}
=&\
  R_0V_cR_0V_cR_0V_c\F
-
  R_0E_c\ord{2}R_0V_c\F
\\
  \Y\ord{4}
=&\
  R_0V_cR_0V_cR_0V_cR_0V_c\F
-
  R_0E_c\ord{3}R_0V_c\F
-
  R_0E_c\ord{2}R_0V_cR_0V_c\F
-
  R_0V_cR_0E_c\ord{2}R_0V_c\F
\\
  \Y\ord{5}
=&\
  R_0V_cR_0V_cR_0V_cR_0V_cR_0V_c\F
-
  R_0E_c\ord{4}R_0V_c\F
+
  R_0E_c\ord{2}R_0E_c\ord{2}R_0V_c\F
-
  R_0E_c\ord{3}R_0V_cR_0V_c\F
\\-&\
  R_0V_cR_0E_c\ord{3}R_0V_c\F
-
  R_0E_c\ord{2}R_0V_cR_0V_cR_0V_c\F
-
  R_0V_cR_0E_c\ord{2}R_0V_cR_0V_c\F
-
  R_0V_cR_0V_cR_0E_c\ord{2}R_0V_c\F
\end{align*}
Projecting these equations by $\br{\F}V_c$ then yields $E_c\ord{2}$, $E_c\ord{3}$, $E_c\ord{4}$, $E_c\ord{5}$, and $E_c\ord{6}$.
\end{ex}

\begin{thm}
\thmtitle{The Bracketing Theorem}
\thmstatement{
The $n\eth$-order contribution to the wavefunction consists of a principal term $(R_0V_c)^n\F=R_0V_c\cd R_0V_c\F$ plus the sum over all possible ways of inserting one or more pairs of brackets $\ip{\cd}\equiv\ip{\F|\cd|\F}$ into the principal term, $R_0V_c\cd R_0\ip{V_c\cd R_0V_c}\cd R_0V_c\F$, allowing nested brackets.
Each of these terms gets a phase factor $(-)^k$ where $k$ is the total number of brackets.
}
\thmproof{
  This obviously holds for $\Y\ord{1}$ since $\Y\ord{1}=R_0V_c\F$ and there are no possible bracketings.
  Assume it holds up to $n-1$ and consider $n$.
  By the substitution lemma, $\Y\ord{n}$ equals a principal term $R_0V_c\cd R_0V_c\F$ plus all unique substitutions of factors $(R_0V_c)^{r_1},\ld,(R_0V_c)^{r_m}$ in the principal term with energy factors $R_0E_c\ord{r_1},\ld,R_0E_c\ord{r_m}$, weighted by a sign $(-)^m$.
  But, by our inductive assumption, the substituted energies $E_c\ord{r_k}=\ip{\F|V_c|\Y\ord{r_k}}$ are sums of a principal term $\ip{V_cR_0V_c\cd R_0V_c}$ plus all possible bracketings, with the appropriate sign factor, which shows that $\Y\ord{n}$ is the sum over all nested bracketings and completes the proof.
}
\end{thm}

\begin{ex}
Noting that bracketings of the form $R_0\ip{V_c}$ vanish because $\ip{V}_c=E_c\ord{1}=0$, and that any bracketing including the last factor vanishes because $R_0\ip{V_c\cd R_0V_c}\F=\ip{V_c\cd R_0V_c}R_0\F=0$, we can write down the bracketing theorem expansion for the first few contributions to the wavefunction as follows.
\begin{align*}
  \Y\ord{1}
=&\
  R_0V_c\F
\\
  \Y\ord{2}
=&\
  R_0V_cR_0V_c\F
\\
  \Y\ord{3}
=&\
  R_0V_cR_0V_cR_0V_c\F
-
  R_0\ip{V_cR_0V_c}R_0V_c\F
\\
  \Y\ord{4}
=&\
  R_0V_cR_0V_cR_0V_cR_0V_c\F
-
  R_0\ip{V_cR_0V_c}R_0V_cR_0V_c\F
-
  R_0V_cR_0\ip{V_cR_0V_c}R_0V_c\F
-
  R_0\ip{V_cR_0V_cR_0V_c}R_0V_c\F
\\
  \Y\ord{5}
=&\
  R_0V_cR_0V_cR_0V_cR_0V_cR_0V_c\F
-
  R_0\ip{V_cR_0V_c}R_0V_cR_0V_cR_0V_c\F
-
  R_0V_cR_0\ip{V_cR_0V_c}R_0V_cR_0V_c\F
\\&\
-
  R_0V_cR_0V_cR_0\ip{V_cR_0V_c}R_0V_c\F
+
  R_0\ip{V_cR_0V_c}R_0\ip{V_cR_0V_c}R_0V_c\F
-
  R_0\ip{V_cR_0V_cR_0V_c}R_0V_cR_0V_c\F
\\&\
-
  R_0V_cR_0\ip{V_cR_0V_cR_0V_c}R_0V_c\F
-
  R_0\ip{V_cR_0V_cR_0V_cR_0V_c}R_0V_c\F
+
  R_0\ip{V_cR_0\ip{V_cR_0V_c}R_0V_c}R_0V_c\F
\end{align*}
The bracketing expansions for the corresponding energies are as follows.
\begin{align*}
  E_c\ord{2}
=&\
  \ip{V_cR_0V_c}
\\
  E_c\ord{3}
=&\
  \ip{V_cR_0V_cR_0V_c}
\\
  E_c\ord{3}
=&\
  \ip{V_cR_0V_cR_0V_cR_0V_c}
-
  \ip{V_cR_0\ip{V_cR_0V_c}R_0V_c}
\\
  E_c\ord{4}
=&\
  \ip{V_cR_0V_cR_0V_cR_0V_cR_0V_c}
-
  \ip{V_cR_0\ip{V_cR_0V_c}R_0V_cR_0V_c}
-
  \ip{V_cR_0V_cR_0\ip{V_cR_0V_c}R_0V_c}
-
  \ip{V_cR_0\ip{V_cR_0V_cR_0V_c}R_0V_c}
\\
  E_c\ord{5}
=&\
  \ip{V_cR_0V_cR_0V_cR_0V_cR_0V_cR_0V_c}
-
  \ip{V_cR_0\ip{V_cR_0V_c}R_0V_cR_0V_cR_0V_c}
-
  \ip{V_cR_0V_cR_0\ip{V_cR_0V_c}R_0V_cR_0V_c}
\\-&\
  \ip{V_cR_0V_cR_0V_cR_0\ip{V_cR_0V_c}R_0V_c}
+
  \ip{V_cR_0\ip{V_cR_0V_c}R_0\ip{V_cR_0V_c}R_0V_c}
-
  \ip{V_cR_0\ip{V_cR_0V_cR_0V_c}R_0V_cR_0V_c}
\\-&\
  \ip{V_cR_0V_cR_0\ip{V_cR_0V_cR_0V_c}R_0V_c}
-
  \ip{V_cR_0\ip{V_cR_0V_cR_0V_cR_0V_c}R_0V_c}
+
  \ip{V_cR_0\ip{V_cR_0\ip{V_cR_0V_c}R_0V_c}R_0V_c}
\end{align*}
\end{ex}

\begin{rmk}
Individual terms in the perturbation expansion are readily evaluated using Wick's theorem and $\F$-normal ordering.
Using resolution of the identity in the orthogonal space, the general structure of a principal term is as follows.
{\footnotesize
\begin{align*}
  (R_0V_c)^n\kt{\F}
=
  Q
  (R_0V_c)^n\kt{\F}
=
  \sum_k
  \kt{\F_k}
  \ip{\F_k|(R_0V_c)^n|\F}
=
  \sum_k
  \kt{\F_k}
  \sum_{k_1\cd k_n}
  \fr{
    \ip{\F_k    |V_c|\F_{k_1}}
    \ip{\F_{k_1}|V_c|\F_{k_2}}
    \ip{\F_{k_2}|\cd|\F_{k_n}}
    \ip{\F_{k_n}|V_c|\F}
  }{
    \mc{D}_{k_1}\mc{D}_{k_2}\cd\mc{D}_{k_n}
  }
\end{align*}}%
Terms with bracketing insertions differ from the principal term by a scalar $\ip{\cd}$ which can be factored out, leaving a squared resolvent at the point of insertion.
By orthonormality of the determinant basis, the squared resolvent equals
\begin{align*}
&&
  R_0^2
=
  \sum_{k_1k_2}
  \fr{
    \kt{\F_{k_1}}
    \ip{\F_{k_1}|\F_{k_2}}
    \br{\F_{k_2}}
  }{
    \mc{D}_{k_1}\mc{D}_{k_2}
  }
=
  \sum_{k_1k_2}
  \fr{
    \kt{\F_{k_1}}
    \d_{k_1k_2}
    \br{\F_{k_2}}
  }{
    \mc{D}_{k_1}\mc{D}_{k_2}
  }
=
  \sum_k
  \fr{
    \kt{\F_k}\br{\F_k}
  }{
    \mc{D}_k^2
  }
\end{align*}
which is of course equal to $(H_0^2)^{-1}$ restricted to the orthogonal space, $R_0^2H_0^2=Q$.
Consequently, every wavefunction contribution is proportional to a term of the following generic form
\begin{align*}
  R_0^{p_1}V_cR_0^{p_2}V_c\cd R_0^{p_n}V_c
  \kt{\F}
=
  \sum_k
  \kt{\F_k}
  \ip{\F|
    \tl{a}_k\dg
    R_0^{p_1}
    V_c
    R_0^{p_2}
    V_c
    \cd
    R_0^{p_n}
    V_c
  |\F}
&&
  R_0^p
=
  \sum_k
  \fr{\kt{\F_k}\br{\F_k}}{\mc{D}_k^p}
\end{align*}
which can be evaluated by determining all complete $\F$-normal contractions of
$
  \tl{a}_k\dg
  R_0^{p_1}
  V_c
  R_0^{p_2}
  V_c
  \cd
  R_0^{p_n}
  V_c
$,
where $\tl{a}_k\dg$ is a de-excitation operator
$\tl{a}_k\dg\in\{\tl{a}^{i_1\cd i_n}_{a_1\cd a_n}\,|\,\substack{i_1<\cd<i_n,\\a_1<\cd<a_n}\}$.
\end{rmk}

\begin{rmk}
Note that, when an operator of the form
$
  M_{\mr{o}}
=
  \sum_i
  a_i\kt{\F}m_i\br{\F}a^i
$
or
$
  M_{\mr{v}}
=
  \sum_a
  a^a\kt{\F}m_a\br{\F}a_a
$
is contracted on the left and on the right, we can make the following rearrangement
\begin{align*}
  \sum_i
  \ctr{}{a}{^p}{a}a^pa_i\kt{\F}m_i\br{\F}\ctr{}{a}{^i}{a}a^ia_s
=
  \kt{\F}\br{\F}
  m_p\,
  \ctr{}{a}{^p}{a}a^pa_s
&&
  \sum_a
  \ctr{}{a}{^q}{a}a_qa^a\kt{\F}m_a\br{\F}\ctr{}{a}{_a}{a}a_aa^r
=
  \kt{\F}\br{\F}
  m_q\,
  \ctr{}{a}{_q}{a}a_q a^r
\end{align*}
where the requirement that $m_p$ have a hole index and $m_q$ have a particle index is taken care of by the contractions.
In words, contracting one operator to the left side of $M_o$ and another operator to the right side is equivalent to contracting these operators to each other and freezing out the term from $M_{\mr{o}}$ that matches the left (or, equivalently, the right) index.
This generalizes directly to operators of the form
$
  \sum_{\substack{i_1<\cd<i_k\\a_1<\cd<a_k}}
  \tl{a}_{i_1\cd i_k}^{a_1\cd a_k}
  \kt{\F}M_{i_1\cd i_k}^{a_1\cd a_k}\br{\F}
  \tl{a}^{i_1\cd i_k}_{a_1\cd a_k}
$
and can be used to simplify the contractions of an operator product with an intervening resolvent, $QR_0^nQ'$:
\begin{align*}
&&
  \gno{\ol{\ol{
    Q R_0^n Q'
  }}}
=
  \sum_k
  \fr{
    \gno{\ol{\ol{Q\tl{a}_k}}}
    \kt{\F}\br{\F}
    \gno{\ol{\ol{\tl{a}_k\dg Q'}}}
  }{
    \mc{D}_k^n
  }
=
  \kt{\F}\br{\F}
  \sum_k
  \fr{
    \gno{\ol{\ol{Q\tl{a}_k}}}
    \gno{\ol{\ol{\tl{a}_k\dg Q'}}}
  }{
    \mc{D}_k^n
  }
=
  \kt{\F}\br{\F}
  \gno{\ol{\ol{
    Q
    \,\resolventline{\scriptsize\ \ $R_0^n$}\hspace{1pt}
    Q'
  }}}\,.
\end{align*}
Here, we have introduced the notion of a \textit{resolvent line}
$
\resolventline[0.6]{\scriptsize\ \ $R_0^n$}\hspace{3pt}
$.
Complete contractions through a resolvent line are defined as
\begin{align*}
&&
  \gno{
  \ctr[4]
    {}
    {a}
    {^{p_1}\cd a^{p_k}a_{q_1}\cd a_{q_k}\,a^{r_1}\cd a^{r_k}a_{s_1}\cd }
    {a}
  \ctr[3]
    {a^{p_1}\cd}
    {a}
    {^{p_k}a_{q_1}\cd a_{q_k}\,a^{r_1}\cd a^{r_k}}
    {a}
  \ctr[2]
    {a^{p_1}\cd a^{p_k}}
    {a}
    {_{q_1}\cd a_{q_k}\,a^{r_1}\cd}
    {a}
  \ctr[1]
    {a^{p_1}\cd a^{p_k}a_{q_1}\cd}
    {a}
    {_{q_k}\,}
    {a}
  a^{p_1}\cd a^{p_k}
  a_{q_1}\cd a_{q_k}
  \resolventline[1.3]{\scriptsize\ \ $R_0^n$}\,
  a^{r_1}\cd a^{r_k}
  a_{s_1}\cd a_{s_k}
  }
\equiv
  \fr{\gno{
    \ctr[4]
      {}
      {a}
      {^{p_1}\cd a^{p_k}a_{q_1}\cd a_{q_k}a^{r_1}\cd a^{r_k}a_{s_1}\cd }
      {a}
    \ctr[3]
      {a^{p_1}\cd}
      {a}
      {^{p_k}a_{q_1}\cd a_{q_k}a^{r_1}\cd a^{r_k}}
      {a}
    \ctr[2]
      {a^{p_1}\cd a^{p_k}}
      {a}
      {_{q_1}\cd a_{q_k}a^{r_1}\cd}
      {a}
    \ctr[1]
      {a^{p_1}\cd a^{p_k}a_{q_1}\cd}
      {a}
      {_{q_k}}
      {a}
    a^{p_1}\cd a^{p_k}
    a_{q_1}\cd a_{q_k}
    a^{r_1}\cd a^{r_k}
    a_{s_1}\cd a_{s_k}
  }}{
    (\mc{D}_{p_1\cd p_k}^{q_1\cd q_k})^n
  }\,.
\end{align*}
That is, each hole contraction $\ctr{}{a}{^p}{a}a^pa_s$ that passes through the resolvent line fixes a hole index in the denominator $(\mc{D}_{\cd p\cd})^n$, and each particle contraction $\ctr{}{a}{_q}{a}a_qa^r$ fixes a particle index $(\mc{D}^{\cd q\cd})^n$.
This result further generalizes to completely contracted products with multiple resolvents, and is codified in the next proposition.
\end{rmk}


\begin{prop}\label{prop:reduced-wick-thm}
\thmtitle{Reduced Wick theorem for RSPT}
\thmstatement{
For $\F$-normal excitation operators $Q$, $Q_1$, $\cd$, $Q_n$, we have
\vspace{8pt}
\begin{align*}
&&
\ip{\F|
  Q
  R_0^{p_1}
  Q_1
  R_0^{p_2}
  Q_2
  \cd
  R_0^{p_n}
  Q_n
|\F}
=
\gno{\ol{\ol{
  Q
  \hspace{2pt}\resolventline{\scriptsize\ \ $R_0^{p_1}$}\hspace{2pt}
  Q_1
  \hspace{2pt}\resolventline{\scriptsize\ \ $R_0^{p_2}$}\hspace{2pt}
  Q_2
  \cd
  \hspace{2pt}\resolventline{\scriptsize\ \ $R_0^{p_n}$}\hspace{2pt}
  Q_n
}}}\,.
\end{align*}%
  That is, inserting a resolvent between the operators in
  $
  \ip{\F|QQ_1\cd Q_n|\F}
  $
  scales the weight of each contraction in the Wick expansion
  $
  \gno{\ol{\ol{QQ_1\cd Q_n}}}
  $
  by a denominator factor $\mc{D}^{q_1\cd q_k}_{p_1\cd p_k}$ where $q_1,\ld, q_k$ label the particle contractions and $p_1,\ld, p_k$ label the hole contractions passing over the point of insertion.
}
\end{prop}

\begin{ex}
The reduced Wick theorem of \Cref{prop:reduced-wick-thm} can be used to evaluate the following
\vspace{5pt}
\begin{align*}
  \gno{\ol{\ol{
    \tl{a}^{p_1q_1}
          _{r_1s_1}
    \hspace{2pt}\resolventline{\tiny\ \ $R_0$}\hspace{2pt}
    \tl{a}^{p_2q_2}
          _{r_2s_2}
  }}}
=&\
  P^{(p_2/q_2)}_{(r_2/s_2)}
  \gno{
    \tl{a}^{p_1^\hole q_1^{\hole\hole}}
          _{r_1^{\ptcl}s_1^{\ptcl\ptcl}}
    \hspace{2pt}\resolventline[0.8]{\tiny\ \ $R_0$}\hspace{2pt}
    \tl{a}^{p_2^{\ptcl}q_2^{\ptcl\ptcl}}
          _{r_2^\hole s_2^{\hole\hole}}
  }
\\[8pt]
  \gno{\ol{\ol{
    \tl{a}^{p_1q_1}
          _{r_1s_1}
    \hspace{2pt}\resolventline{\tiny\ \ $R_0$}\hspace{2pt}
    \tl{a}^{p_2q_2}
          _{r_2s_2}
    \hspace{2pt}\resolventline{\tiny\ \ $R_0$}\hspace{2pt}
    \tl{a}^{p_3q_3}
          _{r_3s_3}
  }}}
=&\
  P^{(p_1/q_1|p_3/q_3)}_{(r_1/s_1|r_3/s_3)}
  \gno{
    \tl{a}^{p_1^{\hole1} q_1^{\hole2}}
          _{r_1^{\ptcl1} s_1^{\ptcl2}}
    \hspace{2pt}\resolventline[0.8]{\tiny\ \ $R_0$}\hspace{2pt}
    \tl{a}^{p_2^{\ptcl1} q_2^{\ptcl2}}
          _{r_2^{\ptcl3} s_2^{\ptcl4}}
    \hspace{2pt}\resolventline[0.8]{\tiny\ \ $R_0$}\hspace{2pt}
    \tl{a}^{p_3^{\ptcl3} q_3^{\ptcl4}}
          _{r_3^{\hole1} s_3^{\hole2}}
  }
+
  P^{(p_1/q_1|p_2/q_2|p_3/q_3)}_{(r_1/s_1|r_2/s_2|r_3/s_3)}
  \gno{
    \tl{a}^{p_1^{\hole3} q_1^{\hole1}}
          _{r_1^{\ptcl3} s_1^{\ptcl1}}
    \hspace{2pt}\resolventline[0.8]{\tiny\ \ $R_0$}\hspace{2pt}
    \tl{a}^{p_2^{\ptcl1} q_2^{\hole2}}
          _{r_2^{\hole1} s_2^{\ptcl2}}
    \hspace{2pt}\resolventline[0.8]{\tiny\ \ $R_0$}\hspace{2pt}
    \tl{a}^{p_3^{\ptcl2} q_3^{\ptcl3}}
          _{r_3^{\hole2} s_3^{\hole3}}
  }
\\[5pt]+&\
  P^{(p_1/q_1|p_3/q_3)}_{(r_1/s_1|r_3/s_3)}
  \gno{
    \tl{a}^{p_1^{\hole3} q_1^{\hole4}}
          _{r_1^{\ptcl1} s_1^{\ptcl2}}
    \hspace{2pt}\resolventline[0.8]{\tiny\ \ $R_0$}\hspace{2pt}
    \tl{a}^{p_2^{\hole1} q_2^{\hole2}}
          _{r_2^{\hole3} s_2^{\hole4}}
    \hspace{2pt}\resolventline[0.8]{\tiny\ \ $R_0$}\hspace{2pt}
    \tl{a}^{p_3^{\ptcl1} q_3^{\ptcl2}}
          _{r_3^{\hole1} s_3^{\hole2}}
  }
\end{align*}
in order to derive the second- and third-order energy contributions
\begin{align*}
  E_c\ord{2}
=&\
  \ip{V_cR_0V_c}
=
  \gno{\ol{\ol{
    V_c
    \hspace{2pt}\resolventline{\tiny\ \ $R_0$}\hspace{2pt}
    V_c
  }}}
=
  \pr{\tfr{1}{4}}^2
  \ol{g}_{p_1q_1}^{r_1s_1}
  \ol{g}_{p_2q_2}^{r_2s_2}
  \gno{\ol{\ol{
    \tl{a}^{p_1q_1}
          _{r_1s_1}
    \hspace{2pt}\resolventline{\tiny\ \ $R_0$}\hspace{2pt}
    \tl{a}^{p_2q_2}
          _{r_2s_2}
  }}}
=
  \fr{1}{4}
  \fr{
    \ol{g}_{ij}^{ab}
    \ol{g}_{ab}^{ij}
  }{
    \mc{D}_{ij}^{ab}
  }
\\[5pt]
  E_c\ord{3}
=&\
  \ip{V_cR_0V_cR_0V_c}
=
  \gno{\ol{\ol{
    V_c
    \hspace{2pt}\resolventline{\tiny\ \ $R_0$}\hspace{2pt}
    V_c
    \hspace{2pt}\resolventline{\tiny\ \ $R_0$}\hspace{2pt}
    V_c
  }}}
=
  \pr{\tfr{1}{4}}^3
  \ol{g}_{p_1q_1}^{r_1s_1}
  \ol{g}_{p_2q_2}^{r_2s_2}
  \ol{g}_{p_3q_3}^{r_3s_3}
  \gno{\ol{\ol{
    \tl{a}^{p_1q_1}
          _{r_1s_1}
    \hspace{2pt}\resolventline{\tiny\ \ $R_0$}\hspace{2pt}
    \tl{a}^{p_2q_2}
          _{r_2s_2}
    \hspace{2pt}\resolventline{\tiny\ \ $R_0$}\hspace{2pt}
    \tl{a}^{p_3q_3}
          _{r_3s_3}
  }}}
=
  \tfr{1}{4}
  \fr{
    \ol{g}_{ij}^{ab}
    \ol{g}_{ab}^{cd}
    \ol{g}_{cd}^{ij}
  }{
    \mc{D}_{ij}^{ab}
    \mc{D}_{ij}^{cd}
  }
+
  \fr{
    \ol{g}_{ij}^{ab}
    \ol{g}_{bk}^{jc}
    \ol{g}_{ca}^{ki}
  }{
    \mc{D}_{ij}^{ab}
    \mc{D}_{ki}^{ca}
  }
+
  \tfr{1}{4}
  \fr{
    \ol{g}_{ij}^{ab}
    \ol{g}_{kl}^{ij}
    \ol{g}_{ab}^{kl}
  }{
    \mc{D}_{ij}^{ab}
    \mc{D}_{kl}^{ab}
  }
\end{align*}
where we have used implicit summation (over all indices) to keep the expressions relatively compact.
Here, we have assumed a canonical Hartree-Fock (RHF or UHF) reference so that the one-particle perturbation vanishes $f_p^q(1-\d_p^q)=0$ and $V_c=\tfr{1}{4}\ol{g}_{pq}^{rs}\tl{a}^{pq}_{rs}$.
Non-canonical Hartree-Fock (such as ROHF) has the same second-order contribution, but introduces two additional terms at third order.
In Hugenholtz diagram notation, this looks as follows.
\begin{align*}
  E_c\ord{2}
=
\diagram[top,bottom]{
  \node[ddot=white] (g1) at (0,+0.4) {};
  \draw[thick,flexdotted] (-0.5,0) to (0.5,0);
  \node[ddot=white] (g2) at (0,-0.4) {};
  \draw[->-] (g1) to ++(-0.25,+0.25);
  \draw[->-] (g1) to ++(+0.25,+0.25);
  \draw[-<-] (g1) to ++(-0.25,-0.25);
  \draw[-<-] (g1) to ++(+0.25,-0.25);
  \draw[->-] (g2) to ++(-0.25,+0.25);
  \draw[->-] (g2) to ++(+0.25,+0.25);
  \draw[-<-] (g2) to ++(-0.25,-0.25);
  \draw[-<-] (g2) to ++(+0.25,-0.25);
}
=
\diagram{
  \node[ddot] (g1) at (0,+0.5) {};
  \draw[thick,flexdotted] (-0.7,0) to ++(1.4,0);
  \node[ddot] (g2) at (0,-0.5) {};
  \draw[->-=0.6] (0,0) [partial ellipse=-90:90:-0.5cm and 0.5cm];
  \draw[-<-=0.6] (0,0) [partial ellipse=-90:90:-0.25cm and 0.5cm];
  \draw[->-=0.6] (0,0) [partial ellipse=-90:90:0.25cm and 0.5cm];
  \draw[-<-=0.6] (0,0) [partial ellipse=-90:90:0.5cm and 0.5cm];
}
&&
  E_c\ord{3}
=
\diagram[top,bottom]{
  \node[ddot=white] (g1) at (0,+0.8) {};
  \draw[thick,flexdotted] (-0.5,+0.4) to ++(1,0);
  \node[ddot=white] (g2) at (0,+0.0) {};
  \draw[thick,flexdotted] (-0.5,-0.4) to ++(1,0);
  \node[ddot=white] (g3) at (0,-0.8) {};
  \draw[->-] (g1) to ++(-0.25,+0.25);
  \draw[->-] (g1) to ++(+0.25,+0.25);
  \draw[-<-] (g1) to ++(-0.25,-0.25);
  \draw[-<-] (g1) to ++(+0.25,-0.25);
  \draw[->-] (g2) to ++(-0.25,+0.25);
  \draw[->-] (g2) to ++(+0.25,+0.25);
  \draw[-<-] (g2) to ++(-0.25,-0.25);
  \draw[-<-] (g2) to ++(+0.25,-0.25);
  \draw[->-] (g3) to ++(-0.25,+0.25);
  \draw[->-] (g3) to ++(+0.25,+0.25);
  \draw[-<-] (g3) to ++(-0.25,-0.25);
  \draw[-<-] (g3) to ++(+0.25,-0.25);
}
=
\diagram{
  \node[ddot] (g1) at (0,+0.8) {};
  \draw[thick,flexdotted] (-1,+0.4) to ++(2,0);
  \node[ddot] (g2) at (0,+0.0) {};
  \draw[thick,flexdotted] (-1,-0.4) to ++(2,0);
  \node[ddot] (g3) at (0,-0.8) {};
  \draw[-<-] (0,0) [partial ellipse=-90:90:-0.8cm and 0.8cm];
  \draw[-<-] (0,0) [partial ellipse=-90:90:+0.8cm and 0.8cm];
  \draw[->-=0.6,bend left =55] (g3) to (g2);
  \draw[->-=0.6,bend right=55] (g3) to (g2);
  \draw[->-=0.6,bend left =55] (g2) to (g1);
  \draw[->-=0.6,bend right=55] (g2) to (g1);
}
+
\diagram{
  \node[ddot] (g1) at (0,+0.8) {};
  \draw[thick,flexdotted] (-1,+0.4) to ++(2,0);
  \node[ddot] (g2) at (0,+0.0) {};
  \draw[thick,flexdotted] (-1,-0.4) to ++(2,0);
  \node[ddot] (g3) at (0,-0.8) {};
  \draw[-<-] (0,0) [partial ellipse=-90:90:-0.8cm and 0.8cm];
  \draw[->-] (0,0) [partial ellipse=-90:90:+0.8cm and 0.8cm];
  \draw[->-=0.6,bend left =55] (g3) to (g2);
  \draw[-<-=0.6,bend right=55] (g3) to (g2);
  \draw[->-=0.6,bend left =55] (g2) to (g1);
  \draw[-<-=0.6,bend right=55] (g2) to (g1);
}
+
\diagram{
  \node[ddot] (g1) at (0,+0.8) {};
  \draw[thick,flexdotted] (-1,+0.4) to ++(2,0);
  \node[ddot] (g2) at (0,+0.0) {};
  \draw[thick,flexdotted] (-1,-0.4) to ++(2,0);
  \node[ddot] (g3) at (0,-0.8) {};
  \draw[->-] (0,0) [partial ellipse=-90:90:-0.8cm and 0.8cm];
  \draw[->-] (0,0) [partial ellipse=-90:90:+0.8cm and 0.8cm];
  \draw[-<-=0.6,bend left =55] (g3) to (g2);
  \draw[-<-=0.6,bend right=55] (g3) to (g2);
  \draw[-<-=0.6,bend left =55] (g2) to (g1);
  \draw[-<-=0.6,bend right=55] (g2) to (g1);
}
\end{align*}
\end{ex}

\begin{ex}\label{ex:second-order-wavefunction-ci-coefficients}
Contributions to the wavefunction can be separated into singles, doubles, triples, etc.~contributions by a resolution of the identity in the orthogonal space
\begin{align*}
  \Y\ord{n}
=
  Q
  \Y\ord{n}
=
  \sum_k
  \pr{\tfr{1}{k!}}^2
  \sum_{\substack{a_1\cd a_k\\i_1\cd i_k}}
  \F_{i_1\cd i_k}^{a_1\cd a_k}\,
  {}^{(n)}c_{a_1\cd a_k}^{i_1\cd i_k}
&&
  {}^{(n)}c_{a_1\cd a_k}^{i_1\cd i_k}
\equiv
  \ip{\F|a_{a_1\cd a_k}^{i_1\cd i_k}|\Y\ord{n}}
\end{align*}
which turns the problem into one of identitfying contributions to the CI coefficients.
At first order in the wavefunction, only doubles contributions are non-vanishing.
\begin{align*}
  {}\ord{1}c_a^i
=&\
  \ip{\tl{a}_a^iR_0V_c}
=
  \gno{\ol{\ol{
    \tl{a}_a^i
    \hspace{2pt}\resolventline{\tiny\ \ $R_0$}\hspace{2pt}
    V_c
  }}}
=
  0
&
  {}\ord{1}c_{ab}^{ij}
=&\
  \ip{\tl{a}^{ij}_{ab}R_0V_c}
=
  \gno{\ol{\ol{
    \tl{a}^{ij}_{ab}
    \hspace{2pt}\resolventline{\tiny\ \ $R_0$}\hspace{2pt}
    V_c
  }}}
=
\diagram{
  \newcommand{\ang}{30};
  \node[ddot] (g) at (0,-0.5) {};
  \draw[thick,flexdotted] (-0.8,0) to ++(1.6,0);
  \draw[-<-=0.6] (g) to ++(90-1.5*\ang:1) node[smalldot] {};
  \draw[-<-=0.6] (g) to ++(90-0.5*\ang:1) node[smalldot] {};
  \draw[->-=0.6] (g) to ++(90+0.5*\ang:1) node[smalldot] {};
  \draw[->-=0.6] (g) to ++(90+1.5*\ang:1) node[smalldot] {};
}
=
  \fr{
    \ol{g}_{ab}^{ij}
  }{
    \mc{D}_{ij}^{ab}
  }
\end{align*}
At second order in the wavefunction, singles, triples, and disconnected quadruples are introduced.
\begin{align*}
  \ord{2}c_a^i
=
  \gno{\ol{\ol{
    \tl{a}_a^i
    \hspace{2pt}\resolventline{\tiny\ \ $R_0$}\hspace{2pt}
    V_c
    \hspace{2pt}\resolventline{\tiny\ \ $R_0$}\hspace{2pt}
    V_c
  }}}
=&\
\diagram{
  \node[ddot] (1g) at (0,-0.5) {};
  \draw[-<-] (1g) to ++(-0.5,1) node[smalldot] {};
  \draw[->-=0.25,->-=0.75] (1g) to node[midway,ddot] (2g) {} ++(+0.5,1)
                                   node[smalldot] {};
  \draw[->-,bend left =45] (1g) to (2g);
  \draw[-<-,bend right=45] (1g) to (2g);
  \draw[thick,flexdotted] (-0.6,-0.35) to ++(1.2,0);
  \draw[thick,flexdotted] (-0.6,+0.35) to ++(1.2,0);
}
+
\diagram{
  \node[ddot] (1g) at (0,-0.5) {};
  \draw[->-] (1g) to ++(-0.5,1) node[smalldot] {};
  \draw[-<-=0.25,-<-=0.75] (1g) to node[midway,ddot] (2g) {} ++(0.5,1)
                                   node[smalldot] {};
  \draw[->-,bend left =45] (1g) to (2g);
  \draw[-<-,bend right=45] (1g) to (2g);
  \draw[thick,flexdotted] (-0.6,-0.35) to ++(1.2,0);
  \draw[thick,flexdotted] (-0.6,+0.35) to ++(1.2,0);
}
=
\diagram{
  \interaction{2}{1g}{(0,-0.5)}{ddot}{sawtooth};
  \node[ddot] (2g2) at (1,0) {};
  \draw[-<-] (1g1) to ++(-0.25,+1) node[smalldot] {};
  \draw[->-=0.25,->-=0.75] (1g1) to node[midway,ddot] (2g1) {} ++(+0.25,+1)
                                    node[smalldot] {};
  \draw[sawtooth] (2g1) to (2g2);
  \draw[->-,bend left =45] (1g2) to (2g2);
  \draw[-<-,bend right=45] (1g2) to (2g2);
  \draw[thick,flexdotted] (-0.3,-0.35) to ++(1.6,0);
  \draw[thick,flexdotted] (-0.3,+0.35) to ++(1.6,0);
}
+
\diagram{
  \interaction{2}{1g}{(0,-0.5)}{ddot}{sawtooth};
  \node[ddot] (2g2) at (1,0) {};
  \draw[->-] (1g1) to ++(-0.25,+1) node[smalldot] {};
  \draw[-<-=0.25,-<-=0.75] (1g1) to node[midway,ddot] (2g1) {} ++(+0.25,+1)
                                    node[smalldot] {};
  \draw[sawtooth] (2g1) to (2g2);
  \draw[->-,bend left =45] (1g2) to (2g2);
  \draw[-<-,bend right=45] (1g2) to (2g2);
  \draw[thick,flexdotted] (-0.3,-0.35) to ++(1.6,0);
  \draw[thick,flexdotted] (-0.3,+0.35) to ++(1.6,0);
}
=
  \fr{1}{2}
  \fr{
    \ol{g}_{am}^{ef}
    \ol{g}_{ef}^{im}
  }{
    \mc{D}_i^a
    \mc{D}_{im}^{ef}
  }
-
  \fr{1}{2}
  \fr{
    \ol{g}_{mn}^{ie}
    \ol{g}_{ae}^{mn}
  }{
    \mc{D}_i^a
    \mc{D}_{mn}^{ae}
  }
\\
  {}\ord{2}c_{ab}^{ij}
=
  \gno{\ol{\ol{
    \tl{a}_{ab}^{ij}
    \hspace{2pt}\resolventline{\tiny\ \ $R_0$}\hspace{2pt}
    V_c
    \hspace{2pt}\resolventline{\tiny\ \ $R_0$}\hspace{2pt}
    V_c
  }}}
=&\
\diagram{
  \node[ddot] (1g) at (0,-0.5) {};
  \node[ddot] (2g) at (0,0) {};
  \draw[-<-] (1g) to ++(-0.65,1) node[smalldot] {};
  \draw[-<-] (1g) to ++(+0.65,1) node[smalldot] {};
  \draw[->-=0.65,bend left =30] (1g) to (2g);
  \draw[->-=0.65,bend right=30] (1g) to (2g);
  \draw[->-] (2g) to (-0.25,+0.5) node[smalldot] {};
  \draw[->-] (2g) to (+0.25,+0.5) node[smalldot] {};
  \draw[thick,flexdotted] (-0.75,-0.35) to ++(1.5,0);
  \draw[thick,flexdotted] (-0.75,+0.35) to ++(1.5,0);
}
+
\diagram{
  \node[ddot] (1g) at (0,-0.5) {};
  \node[ddot] (2g) at (0,0) {};
  \draw[-<-] (1g) to ++(-0.65,1) node[smalldot] {};
  \draw[->-] (1g) to ++(+0.65,1) node[smalldot] {};
  \draw[->-=0.65,bend left =30] (1g) to (2g);
  \draw[-<-=0.65,bend right=30] (1g) to (2g);
  \draw[->-] (2g) to (-0.25,+0.5) node[smalldot] {};
  \draw[-<-] (2g) to (+0.25,+0.5) node[smalldot] {};
  \draw[thick,flexdotted] (-0.75,-0.35) to ++(1.5,0);
  \draw[thick,flexdotted] (-0.75,+0.35) to ++(1.5,0);
}
+
\diagram{
  \node[ddot] (1g) at (0,-0.5) {};
  \node[ddot] (2g) at (0,0) {};
  \draw[->-] (1g) to ++(-0.65,1) node[smalldot] {};
  \draw[->-] (1g) to ++(+0.65,1) node[smalldot] {};
  \draw[-<-=0.65,bend left =30] (1g) to (2g);
  \draw[-<-=0.65,bend right=30] (1g) to (2g);
  \draw[-<-] (2g) to (-0.25,+0.5) node[smalldot] {};
  \draw[-<-] (2g) to (+0.25,+0.5) node[smalldot] {};
  \draw[thick,flexdotted] (-0.75,-0.35) to ++(1.5,0);
  \draw[thick,flexdotted] (-0.75,+0.35) to ++(1.5,0);
}
\\=&\
\diagram{
  \interaction{2}{1g}{(0,-0.5)}{ddot}{sawtooth};
  \draw[-<-] (1g1) to ++(-0.25,1) node[smalldot] {};
  \draw[->-=0.25,->-=0.75] (1g1) to
    node[ddot,midway] (2g1) {} ++(+0.25,1)
    node[smalldot] {};
  \draw[->-=0.25,->-=0.75] (1g2) to
    node[ddot,midway] (2g2) {} ++(-0.25,1)
    node[smalldot] {};
  \draw[-<-] (1g2) to ++(+0.25,1) node[smalldot] {};
  \draw[sawtooth] (2g1) to (2g2);
  \draw[thick,flexdotted] (-0.4,-0.35) to ++(1.8,0);
  \draw[thick,flexdotted] (-0.4,+0.35) to ++(1.8,0);
}
+
\diagram{
  \interaction{2}{1g}{(0,-0.5)}{ddot}{sawtooth};
  \draw[-<-] (1g1) to ++(-0.25,1) node[smalldot] {};
  \draw[->-=0.25,->-=0.75] (1g1) to
    node[ddot,midway] (2g1) {} ++(+0.25,1)
    node[smalldot] {};
  \draw[-<-=0.25,-<-=0.75] (1g2) to
    node[ddot,midway] (2g2) {} ++(-0.25,1)
    node[smalldot] {};
  \draw[->-] (1g2) to ++(+0.25,1) node[smalldot] {};
  \draw[sawtooth] (2g1) to (2g2);
  \draw[thick,flexdotted] (-0.4,-0.35) to ++(1.8,0);
  \draw[thick,flexdotted] (-0.4,+0.35) to ++(1.8,0);
}
+
\diagram{
  \interaction{2}{1g}{(0,-0.5)}{ddot}{sawtooth};
  \draw[->-] (1g1) to ++(-0.25,1) node[smalldot] {};
  \draw[-<-=0.25,-<-=0.75] (1g1) to
    node[ddot,midway] (2g1) {} ++(+0.25,1)
    node[smalldot] {};
  \draw[-<-=0.25,-<-=0.75] (1g2) to
    node[ddot,midway] (2g2) {} ++(-0.25,1)
    node[smalldot] {};
  \draw[->-] (1g2) to ++(+0.25,1) node[smalldot] {};
  \draw[sawtooth] (2g1) to (2g2);
  \draw[thick,flexdotted] (-0.4,-0.35) to ++(1.8,0);
  \draw[thick,flexdotted] (-0.4,+0.35) to ++(1.8,0);
}
=
  \fr{1}{2}
  \fr{
    \ol{g}_{ab}^{ef}
    \ol{g}_{ef}^{ij}
  }{
    \mc{D}_{ij}^{ab}
    \mc{D}_{ij}^{ef}
  }
-
  P^{(i/j)}_{(a/b)}
  \fr{
    \ol{g}_{am}^{ej}
    \ol{g}_{eb}^{im}
  }{
    \mc{D}_{ij}^{ab}
    \mc{D}_{im}^{eb}
  }
+
  \fr{1}{2}
  \fr{
    \ol{g}_{mn}^{ij}
    \ol{g}_{ab}^{mn}
  }{
    \mc{D}_{ij}^{ab}
    \mc{D}_{mn}^{ab}
  }
\\
  {}\ord{2}c_{abc}^{ijk}
=
  \gno{\ol{\ol{
    \tl{a}_{abc}^{ijk}
    \hspace{2pt}\resolventline{\tiny\ \ $R_0$}\hspace{2pt}
    V_c
    \hspace{2pt}\resolventline{\tiny\ \ $R_0$}\hspace{2pt}
    V_c
  }}}
=&\
\diagram{
  \node[ddot] (1g) at (0,-0.5) {};
  \node[ddot] (2g) at (0,0) {};
  \draw[->-,bend left=15] (1g) to (-0.9,0.5) node[smalldot] {};
  \draw[-<-] (1g) to (-0.6,0.5) node[smalldot] {};
  \draw[->-] (1g) to (2g);
  \draw[->-] (2g) to (0,0.5) node[smalldot] {};
  \draw[->-] (2g) to (-0.3,0.5) node[smalldot] {};
  \draw[-<-] (2g) to (+0.3,0.5) node[smalldot] {};
  \draw[-<-] (1g) to (+0.6,0.5) node[smalldot] {};
  \draw[thick,flexdotted] (-1,-0.35) to ++(1.7,0);
  \draw[thick,flexdotted] (-1,+0.35) to ++(1.7,0);
}
+
\diagram{
  \node[ddot] (1g) at (0,-0.5) {};
  \node[ddot] (2g) at (0,0) {};
  \draw[-<-,bend left=15] (1g) to (-0.9,0.5) node[smalldot] {};
  \draw[->-] (1g) to (-0.6,0.5) node[smalldot] {};
  \draw[-<-] (1g) to (2g);
  \draw[-<-] (2g) to (0,0.5) node[smalldot] {};
  \draw[->-] (2g) to (-0.3,0.5) node[smalldot] {};
  \draw[-<-] (2g) to (+0.3,0.5) node[smalldot] {};
  \draw[->-] (1g) to (+0.6,0.5) node[smalldot] {};
  \draw[thick,flexdotted] (-1,-0.35) to ++(1.7,0);
  \draw[thick,flexdotted] (-1,+0.35) to ++(1.7,0);
}
\\=&\
\diagram{
  \interaction{2}{1g}{(0,-0.5)}{ddot}{sawtooth};
  \interaction{2}{2g}{(1.125,0)}{ddot}{sawtooth};
  \draw[-<-] (1g1) to ++(-0.25,1) node[smalldot] {};
  \draw[->-] (1g1) to ++(+0.25,1) node[smalldot] {};
  \draw[-<-] (1g2) to ++(-0.25,1) node[smalldot] {};
  \draw[->-=0.25,->-=0.75] (1g2) to ++(+0.25,1) node[smalldot] {};
  \draw[-<-] (2g2) to ++(-0.25,0.5) node[smalldot] {};
  \draw[->-] (2g2) to ++(+0.25,0.5) node[smalldot] {};
  \draw[thick,flexdotted] (-0.4,-0.35) to ++(2.9,0);
  \draw[thick,flexdotted] (-0.4,+0.35) to ++(2.9,0);
}
+
\diagram{
  \interaction{2}{1g}{(0,-0.5)}{ddot}{sawtooth};
  \interaction{2}{2g}{(1.125,0)}{ddot}{sawtooth};
  \draw[->-] (1g1) to ++(-0.25,1) node[smalldot] {};
  \draw[-<-] (1g1) to ++(+0.25,1) node[smalldot] {};
  \draw[->-] (1g2) to ++(-0.25,1) node[smalldot] {};
  \draw[-<-=0.25,-<-=0.75] (1g2) to ++(+0.25,1) node[smalldot] {};
  \draw[->-] (2g2) to ++(-0.25,0.5) node[smalldot] {};
  \draw[-<-] (2g2) to ++(+0.25,0.5) node[smalldot] {};
  \draw[thick,flexdotted] (-0.4,-0.35) to ++(2.9,0);
  \draw[thick,flexdotted] (-0.4,+0.35) to ++(2.9,0);
}
=
  P^{(ij/k)}_{(a/bc)}
  \fr{
    \ol{g}_{bc}^{ek}
    \ol{g}_{ae}^{ij}
  }{
    \mc{D}_{ijk}^{abc}
    \mc{D}_{ij}^{ae}
  }
-
  P^{(i/jk)}_{(ab/c)}
  \fr{
    \ol{g}_{mc}^{jk}
    \ol{g}_{ab}^{im}
  }{
    \mc{D}_{ijk}^{abc}
    \mc{D}_{im}^{ab}
  }
\\
  {}\ord{2}c_{abcd}^{ijkl}
=
  \gno{\ol{\ol{
    \tl{a}_{abcd}^{ijkl}
    \hspace{2pt}\resolventline{\tiny\ \ $R_0$}\hspace{2pt}
    V_c
    \hspace{2pt}\resolventline{\tiny\ \ $R_0$}\hspace{2pt}
    V_c
  }}}
=&\
\diagram{
  \node[ddot] (1g) at (0,-0.5) {};
  \draw[-<-] (1g) to ++(-0.45,1) node[smalldot] {};
  \draw[-<-] (1g) to ++(-0.15,1) node[smalldot] {};
  \draw[->-] (1g) to ++(+0.15,1) node[smalldot] {};
  \draw[->-] (1g) to ++(+0.45,1) node[smalldot] {};
  \node[ddot] (2g) at (1.2,0) {};
  \draw[-<-] (2g) to ++(-0.45,0.5) node[smalldot] {};
  \draw[-<-] (2g) to ++(-0.15,0.5) node[smalldot] {};
  \draw[->-] (2g) to ++(+0.15,0.5) node[smalldot] {};
  \draw[->-] (2g) to ++(+0.45,0.5) node[smalldot] {};
  \draw[thick,flexdotted] (-0.5,-0.35) to ++(2.2,0);
  \draw[thick,flexdotted] (-0.5,+0.35) to ++(2.2,0);
}
=
\diagram{
  \interaction{2}{1g}{(0,-0.5)}{ddot}{sawtooth};
  \interaction{2}{2g}{(2,0)}{ddot}{sawtooth};
  \draw[-<-] (1g1) to ++(-0.25,1) node[smalldot] {};
  \draw[->-] (1g1) to ++(+0.25,1) node[smalldot] {};
  \draw[-<-] (1g2) to ++(-0.25,1) node[smalldot] {};
  \draw[->-] (1g2) to ++(+0.25,1) node[smalldot] {};
  \draw[-<-] (2g1) to ++(-0.25,0.5) node[smalldot] {};
  \draw[->-] (2g1) to ++(+0.25,0.5) node[smalldot] {};
  \draw[-<-] (2g2) to ++(-0.25,0.5) node[smalldot] {};
  \draw[->-] (2g2) to ++(+0.25,0.5) node[smalldot] {};
  \draw[thick,flexdotted] (-0.4,-0.35) to ++(3.9,0);
  \draw[thick,flexdotted] (-0.4,+0.35) to ++(3.9,0);
}
=
  P^{(ij/kl)}_{(ab/cd)}
  \fr{
    \ol{g}_{cd}^{kl}\ol{g}_{ab}^{ij}
  }{
    \mc{D}_{ijkl}^{abcd}
    \mc{D}_{ij}^{ab}
  }
\end{align*}
Here, the indices $m,n,e,f$ are implicitly summed over, and the Hugenholtz diagrams have been translated into Goldstone diagrams in order to evaluate the phase of each term.
The disconnected quadruples term can be factored as follows
\begin{align*}
  P^{(ij/kl)}_{(ab/cd)}
  \fr{
    \ol{g}_{cd}^{kl}\ol{g}_{ab}^{ij}
  }{
    \mc{D}_{ijkl}^{abcd}
    \mc{D}_{ij}^{ab}
  }
=
  \fr{1}{2}
  P^{(ij/kl)}_{(ab/cd)}
  \pr{
    \fr{
      \ol{g}_{cd}^{kl}\ol{g}_{ab}^{ij}
    }{
      \mc{D}_{ijkl}^{abcd}
      \mc{D}_{ij}^{ab}
    }
  +
    \fr{
      \ol{g}_{ab}^{ij}\ol{g}_{cd}^{kl}
    }{
      \mc{D}_{ijkl}^{abcd}
      \mc{D}_{kl}^{cd}
    }
  }
=
  \fr{1}{2}
  P^{(ij/kl)}_{(ab/cd)}
  \pr{
    \fr{
      \ol{g}_{cd}^{kl}\ol{g}_{ab}^{ij}
      \pr{\mc{D}_{kl}^{cd} + \mc{D}_{ij}^{ab}}
    }{
      \mc{D}_{ijkl}^{abcd}
      \mc{D}_{ij}^{ab}
      \mc{D}_{kl}^{cd}
    }
  }
=
  \fr{1}{2}
  P^{(ij/kl)}_{(ab/cd)}
  {}\ord{1}c_{ab}^{ij}
  {}\ord{1}c_{cd}^{kl}
\end{align*}
where we have combined fractions and used $\mc{D}_{ijkl}^{abcd}=\mc{D}_{ij}^{ab}+\mc{D}_{kl}^{cd}$.
Plugging this result into the CI wave operator, we find that the lowest-order contribution to the quadruple excitation operator is a product of two double excitation operators.
\begin{align*}
  \ord{2}\op{C}_4
=
  \pr{\tfr{1}{4!}}^2
  {}\ord{2}c_{abcd}^{ijkl}
  \tl{a}_{ijkl}^{abcd}
=
  \tfr{1}{2}
  \pr{\tfr{1}{4!}}^2
  \pr{
    P^{(ij/kl)}_{(ab/cd)}
    {}\ord{1}c_{ab}^{ij}
    {}\ord{1}c_{cd}^{kl}
  }
  \tl{a}_{ijkl}^{abcd}
=
  \tfr{1}{2}
  \pr{\tfr{1}{4}{}\ord{1}c_{ab}^{ij}\tl{a}_{ij}^{ab}}
  \pr{\tfr{1}{4}{}\ord{1}c_{cd}^{kl}\tl{a}_{kl}^{cd}}
=
  \tfr{1}{2}
  {}\ord{1}\op{C}_2^2
\end{align*}
\end{ex}

\begin{dfn}
Diagrams with external lines are termed \textit{open}, whereas those with only internal lines are \textit{closed}.
A diagram in which there is a path between any two operators is termed \textit{connected}.
Otherwise, the diagram is \textit{disconnected} and comprises one or more parts without cross-contractions.
A disconnected diagram is considered \textit{linked} if none of its parts is closed.
A disconnected diagram with a closed part is \textit{unlinked}.
\end{dfn}

\begin{dfn}
In perturbation theory, there is one more distinction to be made, owing to the presence of resolvent lines.
The parts of a disconnected diagram are termed \textit{separate} if they each have their own set of resolvent lines.
On the other hand, if some disconnected parts of a diagram share one or more resolvent lines, we consider the parts to make up one \textit{combined} diagram. 
The unliked diagrams arising from bracketed terms form a subcategory of separated diagrams which are termed \textit{insertion diagrams}.
Insertion diagrams have the form
\begin{align*}
&&
  \underset{
    \text{insertion diagram}
  }{\underbrace{
    \cd R_0\ip{V_c\cd R_0V_c}R_0V_c\cd 
  }}
=
  \underset{
    \textit{remainder diagram}
  }{\underbrace{
    \pr{
      \cd R_0R_0V_c\cd
    }
  }}\hspace{5pt}
  \underset{
    \textit{inserted diagram}
  }{\underbrace{
    \ip{V_c\cd R_0V_c}
  }}
\end{align*}
where the \textit{inserted diagram} is closed and the \textit{remainder diagram} may be closed or open, with a squared resolvent at the point of insertion.
\end{dfn}

\begin{thm}\label{thm:factorization}
\thmtitle{The Frantz-Mills Factorization Theorem}
\thmstatement{
A diagram with separate parts that are open at the top and closed at the bottom equals the sum over all relative orderings in the corresponding combined diagram.
}
\thmproof{
  Apart from the resolvent lines, the separate and combined diagrams are identical and share a common numerator, so we need only consider the denominators.
  Let $\mc{D}_1,\ld,\mc{D}_n$ be the denominator factors for one diagram with operators $O_1,\ld,O_n$, ordered from bottom to top,
  and let $\mc{D}_1',\ld,\mc{D}_{n'}'$ and $O_1',\ld,O_{n'}'$ be the denominators and operators of another diagram.
  We proceed by induction on $n+n'=m$, starting with $m=2$ as a base case.
  The only non-trivial possibility for $m=2$ is $n=n'=1$, in which there are two possible orderings for the combined diagram: $O_1$ above $O_1'$ or vice versa.
  Either way the upper resolvent line generates a denominator factor $\mc{D}_1+\mc{D}_1'$, but in the first case the factor from the lower resolvent is $\mc{D}_1'$ and in the second case it is $\mc{D}_1$.
  The sum of the denominators for these two orderings is
$
  \fr{
    1
  }{
    \mc{D}_1(\mc{D}_1+\mc{D}_1')
  }
+
  \fr{
    1
  }{
    \mc{D}_1'(\mc{D}_1+\mc{D}_1')
  }
=
  \fr{
    \mc{D}_1+\mc{D}_1'
  }{
    \mc{D}_1(\mc{D}_1+\mc{D}_1')\mc{D}_1'
  }
=
  \fr{
    1
  }{
    \mc{D}_1\mc{D}_1'
  }
$,
  confirming the proposition for $m=2$.
  Now, assume the proposition for $m-1$ and consider $m$.
  The topmost resolvent generates the same denominator factor $\mc{D}_n+\mc{D}_{n'}'$ for all orderings of the combined diagram.
  Factoring this out, we have one set of orderings with $O_n$ as the topmost operator factor which, by our inductive assumption, add up to $\fr{1}{\mc{D}_1\cd\mc{D}_{n-1}\mc{D}_1'\cd\mc{D}_{n'}'}$, while the remaining orderings all have $O_{n'}'$ as the topmost operator and add up to $\fr{1}{\mc{D}_1\cd\mc{D}_n\mc{D}_1'\cd\mc{D}_{n'-1}'}$.
  The total sum is therefore
$
  \fr{1}{\mc{D}_n+\mc{D}_{n'}'}
  (
    \fr{
      1
    }{
      \mc{D}_1\cd \mc{D}_{n-1}\mc{D}_1'\cd\mc{D}_{n'}'
    }
  +
    \fr{
      1
    }{
      \mc{D}_1\cd \mc{D}_{n}\mc{D}_1'\cd\mc{D}_{n'-1}'
    }
  )
=
  \fr{
    1
  }{
    \mc{D}_1\cd \mc{D}_n\mc{D}_1'\cd\mc{D}_{n'}'
  }\,,
$
  which proves the proposition for all $m$.
  For multiple separate parts, the proposition still follows by combining one pair at a time, since the set of final orderings is the cartesian product of the pairwise orderings.
}
\end{thm}

\begin{ex}\label{ex:frantz-mills-example}
The equation ${}\ord{2}\op{C}_4=\fr{1}{2}{}\ord{1}\op{C}_2^2$ derived in \Cref{ex:second-order-wavefunction-ci-coefficients} is an example of the Frantz-Mills factorization theorem
{\footnotesize
\begin{align*}
  \tfr{1}{2}
  {}\ord{1}\op{C}_2^2
=
  \tfr{1}{2}
  \pr{
    \tfr{1}{4}
    \fr{
      \ol{g}_{ab}^{ij}
    }{
      \mc{D}_{ij}^{ab}
    }
    \tl{a}_{ij}^{ab}
  }
  \pr{
    \tfr{1}{4}
    \fr{
      \ol{g}_{cd}^{kl}
    }{
      \mc{D}_{kl}^{cd}
    }
    \tl{a}_{kl}^{cd}
  }
=
\diagram{
  \interaction{2}{1g}{(0,-0.5)}{ddot}{sawtooth};
  \draw[-<-] (1g1) to ++(-0.25,1);
  \draw[->-] (1g1) to ++(+0.25,1);
  \draw[-<-] (1g2) to ++(-0.25,1);
  \draw[->-] (1g2) to ++(+0.25,1);
  \draw[thick,flexdotted] (-0.35,0.15) to ++(1.7,0);
}\,\,
\diagram{
  \interaction{2}{1g}{(0,-0.5)}{ddot}{sawtooth};
  \draw[-<-] (1g1) to ++(-0.25,1);
  \draw[->-] (1g1) to ++(+0.25,1);
  \draw[-<-] (1g2) to ++(-0.25,1);
  \draw[->-] (1g2) to ++(+0.25,1);
  \draw[thick,flexdotted] (-0.35,0.15) to ++(1.7,0);
}
=
\diagram{
  \interaction{2}{1g}{(0,-0.5)}{ddot}{sawtooth};
  \interaction{2}{2g}{(2,0)}{ddot}{sawtooth};
  \draw[-<-] (1g1) to ++(-0.25,1);
  \draw[->-] (1g1) to ++(+0.25,1);
  \draw[-<-] (1g2) to ++(-0.25,1);
  \draw[->-] (1g2) to ++(+0.25,1);
  \draw[-<-] (2g1) to ++(-0.25,0.5);
  \draw[->-] (2g1) to ++(+0.25,0.5);
  \draw[-<-] (2g2) to ++(-0.25,0.5);
  \draw[->-] (2g2) to ++(+0.25,0.5);
  \draw[thick,flexdotted] (-0.4,-0.35) to ++(3.9,0);
  \draw[thick,flexdotted] (-0.4,+0.35) to ++(3.9,0);
}
=
  \pr{\tfr{1}{2}}^4
  \fr{
    \ol{g}_{cd}^{kl}
    \ol{g}_{ab}^{ij}
  }{
    \mc{D}_{ijkl}^{abcd}
    \mc{D}_{ij}^{ab}
  }
  \tl{a}_{ijkl}^{abcd}
=
  {}\ord{2}\op{C}_4
\end{align*}}%
where we have only one diagrammatically unique ordering for the combined diagram.
If instead we had two different operators $\op{V}$ and $\op{W}$, this equation would instead read as follows.
\begin{align*}
&&
\diagram{
  \interaction{2}{1g}{(0,-0.5)}{ddot}{dashed};
  \draw[-<-] (1g1) to ++(-0.25,1);
  \draw[->-] (1g1) to ++(+0.25,1);
  \draw[-<-] (1g2) to ++(-0.25,1);
  \draw[->-] (1g2) to ++(+0.25,1);
  \draw[thick,flexdotted] (-0.35,0.15) to ++(1.7,0);
}\,\,
\diagram{
  \interaction{2}{1g}{(0,-0.5)}{ddot}{wiggly};
  \draw[-<-] (1g1) to ++(-0.25,1);
  \draw[->-] (1g1) to ++(+0.25,1);
  \draw[-<-] (1g2) to ++(-0.25,1);
  \draw[->-] (1g2) to ++(+0.25,1);
  \draw[thick,flexdotted] (-0.35,0.15) to ++(1.7,0);
}
=&\
\diagram{
  \interaction{2}{1g}{(0,-0.5)}{ddot}{dashed};
  \interaction{2}{2g}{(2,0)}{ddot}{wiggly};
  \draw[-<-] (1g1) to ++(-0.25,1);
  \draw[->-] (1g1) to ++(+0.25,1);
  \draw[-<-] (1g2) to ++(-0.25,1);
  \draw[->-] (1g2) to ++(+0.25,1);
  \draw[-<-] (2g1) to ++(-0.25,0.5);
  \draw[->-] (2g1) to ++(+0.25,0.5);
  \draw[-<-] (2g2) to ++(-0.25,0.5);
  \draw[->-] (2g2) to ++(+0.25,0.5);
  \draw[thick,flexdotted] (-0.4,-0.35) to ++(3.9,0);
  \draw[thick,flexdotted] (-0.4,+0.35) to ++(3.9,0);
}
+
\diagram{
  \interaction{2}{1g}{(0,0)}{ddot}{dashed};
  \interaction{2}{2g}{(2,-0.5)}{ddot}{wiggly};
  \draw[-<-] (1g1) to ++(-0.25,0.5);
  \draw[->-] (1g1) to ++(+0.25,0.5);
  \draw[-<-] (1g2) to ++(-0.25,0.5);
  \draw[->-] (1g2) to ++(+0.25,0.5);
  \draw[-<-] (2g1) to ++(-0.25,1);
  \draw[->-] (2g1) to ++(+0.25,1);
  \draw[-<-] (2g2) to ++(-0.25,1);
  \draw[->-] (2g2) to ++(+0.25,1);
  \draw[thick,flexdotted] (-0.4,-0.35) to ++(3.9,0);
  \draw[thick,flexdotted] (-0.4,+0.35) to ++(3.9,0);
}
\\
&&
\updownarrow&
\\
&&
  \pr{
    \tfr{1}{4}
    \fr{
      v_{ab}^{ij}
    }{
      \mc{D}_{ij}^{ab}
    }
    \tl{a}_{ij}^{ab}
  }
  \pr{
    \tfr{1}{4}
    \fr{
      w_{cd}^{kl}
    }{
      \mc{D}_{kl}^{cd}
    }
    \tl{a}_{kl}^{cd}
  }
=&\
  \pr{\tfr{1}{2}}^4
  \fr{
    w_{cd}^{kl}
    v_{ab}^{ij}
  }{
    \mc{D}_{ijkl}^{abcd}
  }
  \pr{
    \fr{
      1
    }{
      \mc{D}_{ij}^{ab}
    }
  +
    \fr{
      1
    }{
      \mc{D}_{kl}^{cd}
    }
  }
  \tl{a}_{ijkl}^{abcd}
\end{align*}
\end{ex}

\begin{cor}\label{cor:frantz-mills-for-insertion-diagrams}
\thmtitle{The Frantz-Mills Factorization Theorem for Insertion Diagrams}
\thmstatement{
  An insertion diagram equals the sum over all relative orderings in the corresponding combined diagram that keep the top operator of the inserted diagram below the upper resolvent at the point of insertion in the remainder diagram.
}
\thmproof{
  Consider the Frantz-Mills Factorization Theorem for open diagrams as an equation, with the separated diagram on the left and the sum over combined diagrams on the right.
  Notice that the equation still holds true if we attach a feature to the top of any one of the parts, provided we apply this manipulation to each term on both sides of the equation.
  Therefore, we can ``cap'' one of the parts with an operator, add a resolvent line above that, and attach whatever features we want to the top of the other operator.
  This turns the left-hand side into an insertion diagram, and proves the proposition.
}
\end{cor}

\begin{ex}
Closing one of the diagrams in \Cref{ex:frantz-mills-example} and adding a resolvent line, we get the following.
\begin{align*}
&&
\diagram{
  \interaction{2}{1g}{(0,-0.5)}{ddot}{dashed};
  \draw[-<-] (1g1) to ++(-0.25,1);
  \draw[->-] (1g1) to ++(+0.25,1);
  \draw[-<-] (1g2) to ++(-0.25,1);
  \draw[->-] (1g2) to ++(+0.25,1);
  \draw[thick,flexdotted] (-0.35,+0.15) to ++(1.7,0);
  \draw[thick,flexdotted] (-0.35,+0.25) to ++(1.7,0);
}\,\,
\diagram{
  \interaction{2}{1g}{(0,-0.5)}{ddot}{wiggly};
  \interaction{2}{2g}{(0,+0.5)}{ddot}{wiggly};
  \draw[->-,bend left ] (1g1) to (2g1);
  \draw[-<-,bend right] (1g1) to (2g1);
  \draw[->-,bend left ] (1g2) to (2g2);
  \draw[-<-,bend right] (1g2) to (2g2);
  \draw[thick,flexdotted] (-0.35,0.15) to ++(1.7,0);
}
=&\
\diagram{
  \interaction{2}{1g}{(0,-0.6)}{ddot}{dashed};
  \interaction{2}{2g}{(2,-0.25)}{ddot}{wiggly};
  \interaction{2}{3g}{(2,+0.25)}{ddot}{wiggly};
  \draw[-<-] (1g1) to ++(-0.25,1.1);
  \draw[->-] (1g1) to ++(+0.25,1.1);
  \draw[-<-] (1g2) to ++(-0.25,1.1);
  \draw[->-] (1g2) to ++(+0.25,1.1);
  \draw[->-,bend left ] (2g1) to (3g1);
  \draw[-<-,bend right] (2g1) to (3g1);
  \draw[->-,bend left ] (2g2) to (3g2);
  \draw[-<-,bend right] (2g2) to (3g2);
  \draw[thick,flexdotted] (-0.4,-0.4) to ++(3.9,0);
  \draw[thick,flexdotted] (-0.4,+0.0) to ++(3.9,0);
  \draw[thick,flexdotted] (-0.4,+0.4) to ++(3.9,0);
}
+
\diagram{
  \interaction{2}{1g}{(0,-0.25)}{ddot}{dashed};
  \interaction{2}{2g}{(2,-0.6)}{ddot}{wiggly};
  \interaction{2}{3g}{(2,+0.25)}{ddot}{wiggly};
  \draw[-<-] (1g1) to ++(-0.25,0.75);
  \draw[->-] (1g1) to ++(+0.25,0.75);
  \draw[-<-] (1g2) to ++(-0.25,0.75);
  \draw[->-] (1g2) to ++(+0.25,0.75);
  \draw[->-,bend left ] (2g1) to (3g1);
  \draw[-<-,bend right] (2g1) to (3g1);
  \draw[->-,bend left ] (2g2) to (3g2);
  \draw[-<-,bend right] (2g2) to (3g2);
  \draw[thick,flexdotted] (-0.4,-0.4) to ++(3.9,0);
  \draw[thick,flexdotted] (-0.4,+0.0) to ++(3.9,0);
  \draw[thick,flexdotted] (-0.4,+0.4) to ++(3.9,0);
}
\end{align*}
\end{ex}

\begin{rmk}
Recall that an unlinked diagram is a diagram with a disconnected part that is closed.
For the following proof, the term \textit{unlinked part} will be used exclusively to refer to the \textit{closed parts} of a \textit{combined} diagram.
The remaining parts of the combined diagram will be referred to as the \textit{remainder diagram}.
\end{rmk}

\begin{thm}
\thmtitle{The Linked Diagram Theorem}
\thmstatement{
  The $n\eth$ RSPT order wavefunction and energy contributions are given by the linked component of the principal term in the bracketing expansion.
\begin{align*}
&&
  \Y\ord{n}
=
  \pr{
    (R_0V_c)^n\F
  }_{\mr{L}}
&&
  E\ord{n+1}
=
  \ip{
    V_c
    (R_0V_c)^n
  }_{\mr{L}}
\end{align*}
  That is, the bracketed terms exactly cancel with the unlinked component of the principal term.
}
\thmproof{
  If $\Y\ord{n}$ is linked then every unlinked contribution to $V_c\Y\ord{n}$ has an open part and vanishes in the energy expression $E\ord{n+1}=\ip{\F|V_c|\Y\ord{n}}=\ip{\F|V_c|\Y\ord{n}}_{\mr{L}}$.
  Therefore, we restrict our attention to the wavefunction.
  We proceed by induction on the maximum allowed nesting depth $k$ in the bracketing expansion.
  If $k=0$ we have only the principal component.
  Assume it gives rise to at least one diagram with an unlinked part.
  By the nature of the expansion, all other orderings of the parts in this diagram also contribute, and at least one of those orderings places all vertices of the unlinked part adjacent to each other.
  This implies that we could place brackets around those vertices without killing the term, which is a contradiction.
  Therefore, the proposition holds for $k=0$.  
  Now, suppose the proposition holds up to $k-1$ and consider $k$.
  By the energy substitution lemma, we have a sum over all possible $m$-tuple substitutions of adjacent factors $(R_0V_c)^{r}$ in the principale term by $R_0E_c^{(r)}$ times a sign factor $({-})^m$ for the number of substitutions.
  By our inductive assumption, each substituted energy $E_c\ord{r}$ equals $\ip{V_c(R_0V_c)\ord{r-1}}_{\mr{L}}$.
  Then the following lemma proves that these linked bracketings cancel with the unlinked contributions to the principal term, proving the result to all orders.
  \begin{lem}
  \thmstatement{
    Restricting the bracketing expansion of $\Y\ord{n}$ to non-nested insertions and keeping only the linked contributions to each bracket $\ip{\cd}\mapsto\ip{\cd}_{\mr{L}}$ gives the linked component of the principal term, $((R_0V_c)^n\F)_{\mr{L}}$.
}
  \thmproof{
    Suppose the expansion allows at most $k$ brackets.
    Consider one of the terms with $k-1$ brackets, together with the term (if it exists) that shares the same $k-1$ insertions but has an additional insertion below them.
    If there is more than one, pick the term with the lowest insertion.
    Since the two terms differ by one bracket, they are opposite in sign.
    Therefore, by \Cref{cor:frantz-mills-for-insertion-diagrams}, the $k$-bracket term cancels all of the unlinked contributions to the $(k-1)$-bracket term for which the top of the unlinked part is at the level of the insertion.
    Proceeding to the next lowest $k$-bracket term, we successively cancel all unlinked contributions to the $(k-1)$-bracket term below the lowest point of insertion.
    Continuing this procedure, we cancel all $k$-bracket terms against all unlinked contributions below the lowest insertions of the $(k-1)$-bracket terms.
    Note that all unlinked contributions \emph{above} the lowest insertions are still present, which allows us to now cancel against each $(k-2)$-bracket term.    
    Applying this procedure iteratively to the terms with the most brackets, we can eliminate each current set of maximally bracketed terms, until we end up with just the principal term.
  }
  \end{lem}
}
\end{thm}


\end{document}