\documentclass[11pt,fleqn]{article}
\usepackage[cm]{fullpage}
%%AVC PACKAGES
\usepackage{avcgreek}
\usepackage{avcfonts}
\usepackage{avcmath}
\usepackage[numberby=section]{avcthm} % 
\usepackage{qcmacros}
\usepackage{goldstone}
%%MACROS FOR THIS DOCUMENT
\numberwithin{equation}{section}
\usepackage{titlesec}
\titleformat{\section}{\Large\bfseries\mathversion{bold}}{\thesection.}{5pt}{}
% remove header from TOC
\makeatletter
\renewcommand\tableofcontents{%
  \@starttoc{toc}%
}
\makeatother

%%%DOCUMENT%%%
\begin{document}

%%TENSORS
\section{Rayleigh-Schr\"odinger perturbation theory}\label{sec:rspt}

\begin{dfn}\label{dfn:model-hamiltonian}
\thmtitle{Model problem}
In perturbation theory, the \textit{model Hamiltonian} is an operator $H_0\approx H_e$ which approximates the full Hamiltonian and has eigenfunctions spanning the complete $n$-particle Fock space.
An obvious choice in electronic structure is the diagonal one-particle component of $H_e$, which is diagonal in the determinant basis.
\begin{align*}
&&
  H_0
  \F_{i_1\cd i_k}^{a_1\cd a_k}
=
  \mc{E}_{i_1\cd i_k}^{a_1\cd a_k}
  \F_{i_1\cd i_k}^{a_1\cd a_k}
&&
  H_c
=
  H_0
+
  V_c
&&
  H_0
\equiv
  f_p^p\tl{a}^p_p
&&
  \mc{E}_{i_1\cd i_k}^{a_1\cd a_k}
\equiv
\pr{
  f_{a_1}^{a_1}
+
  \cd
+
  f_{a_k}^{a_k}
-
  f_{i_1}^{i_1}
-
  \cd
-
  f_{i_k}^{i_k}
}
\end{align*}
where the perurbation is
$
  V_c
\equiv
  f_p^q(1-\d_p^q)
  \tl{a}^p_q
+
  \tfr{1}{4}
  \ol{g}_{pq}^{rs}
  \tl{a}^{pq}_{rs}
$.
In this context, the reference determinant $\F$ is termed the \textit{model function}.
Note that we are playing fast and loose with Einstein summation here: the $i$ and $a$ indices are not summed over, but the $p$ index in $H_0$ is.
The eigenvalue can be determined by noting that $a^p_p\F_k=n_p^k\F_k$ where $n_p^k$ is the occupation number of $\y_p$ in $\F_k$.
Therefore $\tl{a}^p_p=a^p_p-\tl{a}^{p^\hole}_{p^\hole}=a^p_p - n_p$ which implies that $\tl{a}^p_p\F_k=(n_p^k-n_p)\F_k$.
It then follows that
$
  \sum_p
  f_p^p
  \tl{a}^p_p
  \F_k
=
(
  \sum_{p\in \F_k}
  f_p^p
-
  \sum_{p\in \F}
  f_p^p
)
\F_k
$,
which leads to the expression above for $\mc{E}_{i_1\cd i_k}^{a_1\cd a_k}$.
\end{dfn}


\begin{dfn}\label{dfn:model-function}
\thmtitle{Model space projection operators}
The projection $P=\kt{\F}\br{\F}$ onto the model function is termed the \textit{model space projection operator}, and its orthogonal complement $Q=1-P=\sum_k(\fr{1}{k!})^2\sum_{\substack{a_1\cd a_k\\i_1\cd i_k}}\kt{\F_{i_1\cd i_k}^{a_1\cd a_k}}\br{\F_{i_1\cd i_k}^{a_1\cd a_k}}$ is the \textit{orthogonal space projection operator}.
Note that $P^2=Q^2=1$ and $PQ=QP=0$ are necessary consequences of the fact that $P$ and $Q$ are projection operators for orthogonal subspaces, and note that $P+Q=1$.
Assuming \textit{intermediate normalization}, where we set the norm of the wavefunction such that $\ip{\F|\Y}=1$ rather than $\ip{\Y|\Y}=1$, the model space projection operator takes the wavefunction into our model function, $P\Y=\F\ip{\F|\Y}=\F$.
\end{dfn}

\begin{dfn}
\thmtitle{Resolvent}
Let $R_0$ be minus\footnote{The annoying sign factor is required for consistency with the standard definition $R_0\equiv(E_0-H_0)^{-1}Q$.  Since we have already subtracted off $E_0$, we have $R_0=(-H_0)^{-1}Q$.  This also results in a more convenient sign rule for the bracketing theorem.} the inverse of $H_0$ in the orthogonal space, so that $-R_0H_0=Q$.
The operator $R_0$ is termed the \textit{resolvent}.
Explicitly, we can apply resolution of the identity in the orthogonal space to get
{\footnotesize
\begin{align*}
  R_0
=
  (-H_0)^{-1}Q
=
  \sum_k
  \pr{\fr{1}{k!}}^2
  \sum_{\substack{a_1\cd a_k\\i_1\cd i_k}}
  (-H_0)^{-1}
  \kt{\F_{i_1\cd i_k}^{a_1\cd a_k}}
  \br{\F_{i_1\cd i_k}^{a_1\cd a_k}}
=
  \sum_k
  \pr{\fr{1}{k!}}^2
  \sum_{\substack{a_1\cd a_k\\i_1\cd i_k}}
  \fr{
    \kt{\F_{i_1\cd i_k}^{a_1\cd a_k}}
    \br{\F_{i_1\cd i_k}^{a_1\cd a_k}}
  }{
    \mc{D}_{i_1\cd i_k}^{a_1\cd a_k}
  }
&&
  \mc{D}_{i_1\cd i_k}^{a_1\cd a_k}
\equiv
-
  \mc{E}_{i_1\cd i_k}^{a_1\cd a_k}
\end{align*}}%
where we recognize that $H_0^{-1}$ does not exist outside of the model space because $H_0\F=0\implies H_0P=0$.
Note that $R_0$ simply acts as the null operator outside of the orthogonal space, so that $R_0Q=R_0$ and $R_0P=0$.
\end{dfn}

\begin{dfn}
\thmtitle{Rayleigh-Schr\"odinger perturbation theory}
Rearranging the Schr\"odinger equation to the form
$
  H_0\Y
=
  (E_c - V_c)\Y
$
and operating $R_0$ on both sides, recognizing that $R_0H_0=-Q$ and $Q\Y=(1-P)\Y=\Y-\F$, we find
\begin{align*}
  R_0H_0\Y
=
-
  Q\Y
=
-
  \Y
+
  \F
=
  R_0
  (E_c - V_c)
  \Y
\implies
  \Y
=
  \F
-
  R_0
  (E_c - V_c)
  \Y
=
  \F
+
  R_0
  (V_c - E_c)
  \Y
\end{align*}
which gives a recursive equation for $\Y$.
Straightforward induction gives
$
  \Y
=
  \sum_{k=0}^n
  (R_0(V_c - E_c))^k\F
+
  (R_0(V_c - E_c))^{n+1}\Y
$.
Noting that $H_0\F=0$ and $\ip{\F|\Y}=1$, projecting the Schr\"odinger equation by $\F$ gives an expression for the correlation energy: $E_c=\ip{\F|V_c|\Y}$.
Assuming the recursive definition for $\Y$ converges, we find
\begin{align}\label{eq:rspt-equations-form-1}
  \Y
=
  \sum_{k=0}^{\infty}
  (R_0(V_c - E_c))^k\F
&&
  E_c
=
  \sum_{k=0}^{\infty}
  \ip{\F|V_c|\Y\ord{k}}
=
  \sum_{k=0}^{\infty}
  \ip{\F|V_c(R_0(V_c - E_c))^k|\F}
\end{align}
which can be solved iteratively in orders of perturbation theory.
Introducing a perturbation parameter $V_c\mapsto \la V_c$ that acts as a switch to turn the perturbation on, $\la=1$, or off, $\la=0$, the wavefunction and correlation energy are given by
\begin{align*}
  \Y(\la)
=
  \sum_k
  \fr{1}{k!}\,\la^k\pr{\pd{^k\Y}{\la^k}}_{\la=0}
\equiv
  \sum_k
  \la^k \Y\ord{k}
&&
  E_c(\la)
=
  \sum_k
  \fr{1}{k!}\,\la^k\pr{\pd{^kE_c}{\la^k}}_{\la=0}
\equiv
  \sum_k
  \la^k
  E_c\ord{k}
\end{align*}
and we can separate \Cref{eq:rspt-equations-form-1} in powers of $\la$.
The first-order energy contribution vanishes
$\la E_c\ord{1}=\la\ip{\F|V_c|\F}=0$
since $V_c$ is composed of $\F$-normal operators.
The first order wavefunction contribution is
$
  \la
  \Y\ord{1}
=
  \la
  R_0
  (V_c - E_c\ord{1})\F
=
  \la
  R_0
  V_c
  \F
$,
which can be directly evaluated using Wick's theorem and $\F$ normal ordering
\begin{align*}
  \Y\ord{1}
=
  R_0
  V_c
  \F
=&\
  \sum_k
  \pr{\fr{1}{k!}}^2
  \sum_{\substack{a_1\cd a_k\\i_1\cd i_k}}
  \kt{\F_{i_1\cd i_k}^{a_1\cd a_k}}
  \fr{
    \ip{\F|\tl{a}^{i_1\cd i_k}_{a_1\cd a_k}V_c|\F}
  }{
    \mc{D}_{i_1\cd i_k}^{a_1\cd a_k}
  }
\\=&\
  \sum_{ia}
  \kt{\F_i^a}
  \fr{
    \sum_{pq}
    f_p^q(1-\d_p^q)
    \ip{\F|\gno{\tl{a}_{a^\ptcl}^{i^\hole}\tl{a}^{p^\ptcl}_{q^\hole}}|\F}
  }{
    f_i^i
  -
    f_a^a
  }
+
  \fr{1}{4}
  \sum_{ijab}
  \kt{\F_{ij}^{ab}}
  \fr{
    \fr{1}{4}
    \sum_{pqrs}
    \ol{g}_{pq}^{rs}
    P^{(p/q)}_{(r/s)}
    \ip{\F|\gno{\tl{a}_{a^\ptcl b^{\ptcl\ptcl}}^{i^\hole j^{\hole\hole}}\tl{a}^{p^\ptcl q^{\ptcl\ptcl}}_{r^\hole s^{\hole\hole}}}|\F}
  }{
    f_i^i
  +
    f_j^j
  -
    f_a^a
  -
    f_b^b
  }
\\=&\
  \sum_{ia}
  \kt{\F_i^a}
  \fr{
    f_a^i
  }{
    f_i^i
  -
    f_a^a
  }
+
  \fr{1}{4}
  \sum_{ijab}
  \kt{\F_{ij}^{ab}}
  \fr{
    \ol{g}_{ab}^{ij}
  }{
    f_i^i
  +
    f_j^j
  -
    f_a^a
  -
    f_b^b
  }
\end{align*}
where we have recognized that only singly and doubly excited determinants can fully contract $V_c$.
The second-order energy contribution, $\la^2 E_c\ord{2}=\la^2\ip{\F|V_c|\Y\ord{1}}$, can be evaluated from our expression for $\Y\ord{1}$.
\begin{align*}
  E\ord{2}
=
  \sum_{ia}
  \ip{\F|V_c\tl{a}^a_i|\F}
  \fr{
    f_a^i
  }{
    f_i^i
  -
    f_a^a
  }
+
  \fr{1}{4}
  \sum_{ijab}
  \ip{\F|V_c\tl{a}^{ab}_{ij}|\F}
  \fr{
    \ol{g}_{ab}^{ij}
  }{
    f_i^i
  +
    f_j^j
  -
    f_a^a
  -
    f_b^b
  }
=
  \sum_{ia}
  \fr{
    f_i^a
    f_a^i
  }{
    f_i^i
  -
    f_a^a
  }
+
  \fr{1}{4}
  \sum_{ijab}
  \fr{
    \ol{g}_{ij}^{ab}
    \ol{g}_{ab}^{ij}
  }{
    f_i^i
  +
    f_j^j
  -
    f_a^a
  -
    f_b^b
  }
\end{align*}
The second order wavefunction contribution is
$
  \la^2
  \Y\ord{2}
=
-
  \la^2
  E_c\ord{2}
  R_0\F
+
  \la^2
  R_0(V_c - E_c\ord{1})R_0(V_c - E_c\ord{1})\F
=
  \la^2
  R_0V_cR_0V_c\F
$
since $R_0\F=0$ and $E_c\ord{1}=0$.
The third order energy can be then obtained from $\Y\ord{2}$ as
$\la^3 E_c\ord{3} = \la^3\ip{\F|V_c|\Y\ord{2}}$.
In this manner, one can in principle solve the Schr\"odinger equation recursively by alternately evaluating the wavefunction and energy contributions at increasing orders in the perturbation parameter.
\end{dfn}

\begin{drv}
Writing the RSPT wavefunction equation as $\Y=\sum_{k=0}^{\infty}(R_0V_c - R_0 E_c)^k\F$, note that if $R_0$ and $V_c$ were to commute we could to an ordinary binomial expansion of $(R_0V_c - R_0E_c)^k$ to give $\sum_{p=0}^k{k\choose p}(-)^p(R_0V_c)^{k-p}(R_0E_c)^p$.
Since they don't commute, we can write the binomial expansion in the following slightly modified form
\begin{align*}
  (R_0E_c - R_0V_c)^k
=
  \sum_{p=0}^k
  (-)^p
  \{R_0E_c\}^p\mr{insert}\{R_0V_c\}^{k-p}
\end{align*}
where $\{B_1,\ld,B_p\}\mr{insert}\{A\}^{k-p}$ denotes the sum over all ${k\choose p}$ possible ways of inserting $k-p$ copies of $A$ into the product $B_1\cd B_p$.
For example,
$
  \{B_1,B_2\}\mr{insert}\{A\}^2
$
evaluates to
$
  AAB_1B_2
+
  AB_1AB_2
+
  AB_1B_2A
+
  B_1AAB_2
+
  B_1AB_2A
+
  B_1B_2AA
$.
This allows the wavefunction expansion to be easily grouped by orders
\begin{align*}
  \Y
=&\
  \sum_{k=0}^{\infty}
  \sum_{p=0}^k
  (-)^p
  \{R_0E_c\}^p
  \mr{insert}\{R_0V_c\}^{k-p}
  \F
\\=&\
  \sum_{n=0}^{\infty}
  \sum_{(n_1,n_2)}^{\mc{C}_2(n)\cup\{(0,n)\}}
  \sum^{\mc{C}(n_1)}_{(r_1,\ld,r_m)}
  (-)^m
  \{R_0E_c\ord{r_1},\ld,R_0E_c\ord{r_m}\}
  \mr{insert}
  \{R_0V_c\}^{n_2}
  \F
=
  \sum_{n=0}^\infty
  \Y\ord{n}
\end{align*}
where $\mc{C}(n)$ denotes the set of integer compositions of $n$, i.e. all ordered tuples $(r_1,\ld,r_m)$ of strictly positive integers that add up to $n$.
$\mc{C}_k(n)\subset\mc{C}(n)$ is the set of $k$-tuple integer compositions of $n$, i.e. all $(r_1,\ld,r_k)$ of fixed length $k$ such that $r_1+\cd+r_k=n$.
The rearrangement follows from the fact that all possible terms of the form
$
  (-)^k
  \{R_0E_c\ord{n_1},\ld,R_0E_c\ord{n_k}\}
  \mr{insert}
  \{R_0V_c\}^{n_{k+1}}
  \F
$
contribute to the sum, and the composition sums group these into all possible terms of this form that are of a given order $n$ in the perturbation parameter $\la$.
Note that we have appended the tuple $(0,n)$ to our sum over $\mc{C}_2(n)$ but not $(n,0)$ since $R_0$ acting directly on $\F$ gives $0$.
These results are summarized in \Cref{lem:energy-substitution-lemma}.
\end{drv}

\begin{lem}\label{lem:energy-substitution-lemma}
\thmtitle{The Energy Substitution Lemma}
\thmstatement{
The $n\eth$-order contribution to the wavefunction is given by
\begin{align*}
  \Y\ord{n}
=
  (R_0V_c)^n\F
+
  \sum_{(n_1,n_2)}^{\mc{C}_2(n)}
  \sum^{\mc{C}(n_1)}_{(r_1,\ld,r_m)}
  (-)^m
  \{R_0E_c\ord{r_1},\ld,R_0E_c\ord{r_m}\}
  \mr{insert}
  \{R_0V_c\}^{n_2}
  \F
\end{align*}
which can be evaluated as the sum of a \emph{principal term}, $(R_0V_c)^n\F$, plus all possible $m$-tuple substitutions of adjacent factors $(R_0V_c)^{r_k}$ in the principal term by $R_0E_c\ord{r_k}$ times a sign factor $(-)^m$.
}
\end{lem}

\begin{ex}
Using the energy substitution lemma, we can directly write down the first few wavefunction contributions
\begin{align*}
  \Y\ord{1}
=&\
  R_0V_c\F
\\
  \Y\ord{2}
=&\
  R_0V_cR_0V_c\F
-
  R_0E_c\ord{1}R_0V_c\F
-
  R_0V_cR_0E_c\ord{1}\F
\\
  \Y\ord{3}
=&\
  R_0V_cR_0V_cR_0V_c\F
-
  R_0E_c\ord{2}R_0V_c\F
-
  R_0V_cR_0E_c\ord{2}\F
+
  R_0E_c\ord{1}R_0E_c\ord{1}R_0V_c\F
+
  R_0E_c\ord{1}R_0V_cR_0E_c\ord{1}\F
\\&\
+
  R_0V_cR_0E_c\ord{1}R_0E_c\ord{1}\F
-
  R_0E_c\ord{1}R_0V_cR_0V_c\F
-
  R_0V_cR_0E_c\ord{1}R_0V_c\F
-
  R_0V_cR_0V_cR_0E_c\ord{1}\F
\end{align*}
where we have directly evaluated the formula of \Cref{lem:energy-substitution-lemma} without simplifying.
These expressions can be simplified by recognizing that $E_c\ord{1}=0$ and that any term with an energy factor next to $\F$ vanishes since $R_0\F=0$.
Omitting these terms, the wavefunction contributions can be simplified as follows.
\begin{align*}
  \Y\ord{1}
=&\
  R_0V_c\F
\\
  \Y\ord{2}
=&\
  R_0V_cR_0V_c\F
\\
  \Y\ord{3}
=&\
  R_0V_cR_0V_cR_0V_c\F
-
  R_0E_c\ord{2}R_0V_c\F
\\
  \Y\ord{4}
=&\
  R_0V_cR_0V_cR_0V_cR_0V_c\F
-
  R_0E_c\ord{3}R_0V_c\F
-
  R_0E_c\ord{2}R_0V_cR_0V_c\F
-
  R_0V_cR_0E_c\ord{2}R_0V_c\F
\\
  \Y\ord{5}
=&\
  R_0V_cR_0V_cR_0V_cR_0V_cR_0V_c\F
-
  R_0E_c\ord{4}R_0V_c\F
+
  R_0E_c\ord{2}R_0E_c\ord{2}R_0V_c\F
-
  R_0E_c\ord{3}R_0V_cR_0V_c\F
\\-&\
  R_0V_cR_0E_c\ord{3}R_0V_c\F
-
  R_0E_c\ord{2}R_0V_cR_0V_cR_0V_c\F
-
  R_0V_cR_0E_c\ord{2}R_0V_cR_0V_c\F
-
  R_0V_cR_0V_cR_0E_c\ord{2}R_0V_c\F
\end{align*}
Projecting these equations by $\br{\F}V_c$ then yields $E_c\ord{2}$, $E_c\ord{3}$, $E_c\ord{4}$, $E_c\ord{5}$, and $E_c\ord{6}$.
\end{ex}

\begin{thm}
\thmtitle{The Bracketing Theorem}
\thmstatement{
The $n\eth$-order contribution to the wavefunction is the sum of a principal term $\underset{\text{$n$ times}}{R_0V_c\cd R_0V_c}\F$ plus all unique ways of placing one or more brackets $R_0V_c\cd R_0\ip{V_c\cd R_0V_c}\cd R_0V_c\F$ into the principal term,\footnote{The brackets $\ip{\cd}$ here are $\F$ expectation values, $\ip{\F|\cd|\F}$.} allowing nested brackets.
Each of these terms gets a phase factor $(-)^k$ where $k$ is the number of brackets.
}
\thmproof{
  This obviously holds for $\Y\ord{1}$ since $\Y\ord{1}=R_0V_c\F$ and there are no possible bracketings.
  Assume it holds up to $n-1$ and consider $n$.
  By the substitution lemma, $\Y\ord{n}$ equals a principal term $R_0V_c\cd R_0V_c\F$ plus all unique substitutions of factors $(R_0V_c)^{r_1},\ld,(R_0V_c)^{r_m}$ in the principal term with energy factors $R_0E_c\ord{r_1},\ld,R_0E_c\ord{r_m}$, weighted by a sign $(-)^m$.
  But, by our inductive assumption, the substituted energies $E_c\ord{r_k}=\ip{\F|V_c|\Y\ord{r_k}}$ are sums of a principal term $\ip{V_cR_0V_c\cd R_0V_c}$ plus all possible bracketings, with the appropriate sign factor, which shows that $\Y\ord{n}$ is the sum over all nested bracketings and therefore completes the proof.
}
\end{thm}

\begin{ex}
Noting that bracketings of the form $R_0\ip{V_c}$ vanish because $\ip{V}_c=E_c\ord{1}=0$, and that any bracketing including the last factor vanish because $R_0\ip{V_c\cd R_0V_c}\F=\ip{V_c\cd R_0V_c}R_0\F=0$, we can write down the bracketing theorem expansion for the first few contributions to the wavefunction as follows.
\begin{align*}
  \Y\ord{1}
=&\
  R_0V_c\F
\\
  \Y\ord{2}
=&\
  R_0V_cR_0V_c\F
\\
  \Y\ord{3}
=&\
  R_0V_cR_0V_cR_0V_c\F
-
  R_0\ip{V_cR_0V_c}R_0V_c\F
\\
  \Y\ord{4}
=&\
  R_0V_cR_0V_cR_0V_cR_0V_c\F
-
  R_0\ip{V_cR_0V_c}R_0V_cR_0V_c\F
-
  R_0V_cR_0\ip{V_cR_0V_c}R_0V_c\F
-
  R_0\ip{V_cR_0V_cR_0V_c}R_0V_c\F
\\
  \Y\ord{5}
=&\
  R_0V_cR_0V_cR_0V_cR_0V_cR_0V_c\F
-
  R_0\ip{V_cR_0V_c}R_0V_cR_0V_cR_0V_c\F
-
  R_0V_cR_0\ip{V_cR_0V_c}R_0V_cR_0V_c\F
\\&\
-
  R_0V_cR_0V_cR_0\ip{V_cR_0V_c}R_0V_c\F
+
  R_0\ip{V_cR_0V_c}R_0\ip{V_cR_0V_c}R_0V_c\F
-
  R_0\ip{V_cR_0V_cR_0V_c}R_0V_cR_0V_c\F
\\&\
-
  R_0V_cR_0\ip{V_cR_0V_cR_0V_c}R_0V_c\F
-
  R_0\ip{V_cR_0V_cR_0V_cR_0V_c}R_0V_c\F
+
  R_0\ip{V_cR_0\ip{V_cR_0V_c}R_0V_c}R_0V_c\F
\end{align*}
The bracketing expansions for the corresponding energies are as follows.
\begin{align*}
  E_c\ord{2}
=&\
  \ip{V_cR_0V_c}
\\
  E_c\ord{3}
=&\
  \ip{V_cR_0V_cR_0V_c}
\\
  E_c\ord{3}
=&\
  \ip{V_cR_0V_cR_0V_cR_0V_c}
-
  \ip{V_cR_0\ip{V_cR_0V_c}R_0V_c}
\\
  E_c\ord{4}
=&\
  \ip{V_cR_0V_cR_0V_cR_0V_cR_0V_c}
-
  \ip{V_cR_0\ip{V_cR_0V_c}R_0V_cR_0V_c}
-
  \ip{V_cR_0V_cR_0\ip{V_cR_0V_c}R_0V_c}
-
  \ip{V_cR_0\ip{V_cR_0V_cR_0V_c}R_0V_c}
\\
  E_c\ord{5}
=&\
  \ip{V_cR_0V_cR_0V_cR_0V_cR_0V_cR_0V_c}
-
  \ip{V_cR_0\ip{V_cR_0V_c}R_0V_cR_0V_cR_0V_c}
-
  \ip{V_cR_0V_cR_0\ip{V_cR_0V_c}R_0V_cR_0V_c}
\\-&\
  \ip{V_cR_0V_cR_0V_cR_0\ip{V_cR_0V_c}R_0V_c}
+
  \ip{V_cR_0\ip{V_cR_0V_c}R_0\ip{V_cR_0V_c}R_0V_c}
-
  \ip{V_cR_0\ip{V_cR_0V_cR_0V_c}R_0V_cR_0V_c}
\\-&\
  \ip{V_cR_0V_cR_0\ip{V_cR_0V_cR_0V_c}R_0V_c}
-
  \ip{V_cR_0\ip{V_cR_0V_cR_0V_cR_0V_c}R_0V_c}
+
  \ip{V_cR_0\ip{V_cR_0\ip{V_cR_0V_c}R_0V_c}R_0V_c}
\end{align*}
\end{ex}


\end{document}
